% !TEX root = ../../thesis.tex

\chapter{Conclusion} % (fold)
\label{cha:conclusion}



\section{Contributions of this thesis} % (fold)
\label{sec:rappel_des_contributions}

This thesis started with the desire to study mechanisms that allow open-ended learning and development in robots and humans. In particular, this thesis aimed to study how the morphological properties of the body could impact the acquisition of motor or social skills.
We realized that if one wanted to really study the role of the body in cognition, one needed to be able to consider the body as an experimental variable: something that can be easily changed and experimented with.

Eventually we shifted from this first goal to focus our attention on more pragmatic ones. If we want to study the role of morphology in the real world, we need a real robotic platform whose morphology allows the exploration of morphological variants. Furthermore, a key aspect of Science is the reproducibility of results, so we need also to find appropriate methods to facilitate the evaluation of our work in other laboratories.
Yet, this was impossible at the time because robot platforms were developed using classical machining techniques requiring a lot of time, resources, energy and funding. Also, classical machining techniques did not allow certain shapes to be built. In this thesis, we decided to take advantage of the 3D printing revolution by transposing it to humanoid robotics and this led us to design the Poppy platform.

All aspects of the platform were designed to be highly modular, modifiable, robust, and easily replicable in other labs for cumulative science. In just a few days, we can now systematically study how various shapes of the legs or feet influence balance in biped locomotion, or how various head morphologies will provoke different reactions when socially interacting with humans.

Poppy was firstly presented as an easily hackable platforms in the AMAM2013 conference~\parencite{lapeyre:hal-00788433} while its design was explained in more detail for IROS 2013~\parencite{lapeyre:hal-00852858} as well as its potential relevance for exploring interaction~\parencite{lapeyre:hal-00984312}. Then we conducted several experiments to demonstrate its unique properties. On one hand, by exploring the role of thigh morphology over bipedal dynamics, presented at Humanoids2013~\parencite{lapeyre:hal-00861110}. On the other hand, by demonstrating it is indeed possible to make morphology an experimental variable. Experiments have been done to test various feet designs and will be presented at Humanoids2014~\parencite{lapeyre2014humanoids}.

Secondly, we used open source release, which drew particular attention, to start the creation of a multidisciplinary community. Thanks to the high level of interest we received, we found educational actors and artists eager to explore novel applications with robots. This work led us to conduct several instructive experiments opening perspectives for the use of Poppy for Art and Education; an associated paper was published for DI2014~\parencite{lapeyreDI}.

Finally, the main contribution has certainly been the open source release of the first complete 3D-printed humanoid robot that can be freely used by anyone as an experimental platform. It compensates for the lack of open science tools available, so robotics researchers, even those working with robot morphology, can share their work with the scientific community. Moreover, Poppy offers an alternative to laboratories desiring to experiment in the real world. They are no longer constrained to either buying a closed and limited robot, or investing resources in the development of a new experimental platform, they can choose to use the work already done with Poppy and adapt it to their needs.
Several research labs in Europe have already begun to use the Poppy platform for their own projects (e.g. Collège de France, Bristol Robotics Lab., Inria Nancy…).


\section{Limits} % (fold)

Numerous limitations have already been discussed during this thesis. In this final conclusion, we would like to focus particular attention on the one associated with the initial motivations that led to the design of Poppy: namely creating an open source experimental robot for exploring the role of morphology.

Firstly,  For us, it appears to be even more difficult than designing a whole humanoid robot because we are not in our field of expertise.
Firstly, we did not have the time yet to explore real case of open science where two laboratories are evolved on the same scientific experiment to evaluate how fluent the reproducibility is. Also as we saw in the previous discussion chapter, the creation of a community is really challenging and they are still mediation needed to promote open collaboration in the research community.

Secondly, because we spend many time on the "Poppy open environment", the scientific contributions on the role of morphology were limited to preliminary results. We conducted experiments on the role of the thigh morphology that showed great improvements of a bio-inspired design over a classical straight thigh, yet these results are very limited to a specific case where the robot’s balance is ensured by physical guidance. Also other labs already using Poppy did not yet publish results associated with the use of Poppy as main tool. Yet, Poppy was released open source one year ago, so it requires probably more time to be both adopted and used for research.

Finally, we designed Poppy to explore the role of morphology, especially in the scope of biped locomotion. The design of the platform has been directed toward new ways of achieving biped locomotion. Poppy has small feet, under-actuated legs, a multi-articulated torso and so on. While we are convinced these solutions are more interesting for research purposes, they are still less efficient than classical approaches using big feet and powerful actuators.
Therefore currently, Poppy cannot walk by itself and it is clearly a limitation of the platform in particular for educational applications. To overcome this problem, we will use the modularity of Poppy to offer an alternative version of the leg design with a more traditional configuration, more suitable to achieve quickly a limited yet working biped locomotion.



\section{What is Poppy ?} % (fold)
\label{sec:what_is_poppy}

In this thesis, we suggest a novel approach to creating experimental robot platforms. To implement this methodology we created a set of tools based on open source environment and emergent technologies involving mechanics, electronics, software and a community (under construction). As a first instance, we used these tools to create a humanoid robot but actually, these tools could have been used for any kind of robot that has to move and act in the real world.

In this context, "Poppy" is more a meta-robot than an actual humanoid robot. The Poppy humanoid is an instance of this meta-robot, and thanks to the modularization of our hardware technology in progress, it will be more and more clear than we can easily reshape and reconfigure Poppy into any kind of robot, with various, sizes, number of DoF, limbs and so on. The community tools we are organizing will allow a multidisciplinary community of robotics enthusiasts to create and share new creatures based on the same technological bricks. Of course the challenges of creating a multidisciplinary community and hardware modularity presented in section~\ref{sec:creating_a_multi} and section~\ref{sec:open_sourcing_the_robotic_actuations} will have to be addressed. In this way, the open source actuators we will develop will represent an essential technological brick to foster the creation of these creatures. In addition, special attention will be focused on the educational impact of such creative environment with the construction of open source educational content and ready-to-use projects.


