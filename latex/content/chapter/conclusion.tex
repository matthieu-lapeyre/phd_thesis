% !TEX root = ../../thesis.tex

\chapter{Conclusion} % (fold)
\label{cha:conclusion}

A key-aspect of Science is the results reproducibility but unfortunately in the robotics field, the amount of material resources and the techniques involved makes it difficult to transfer robot platforms from one lab to another. Because Poppy allows cheap and quick reproduction in any labs, and because it is freely distributed under open source licenses, Poppy is intrinsically designed to permit reproducibility.

The Poppy platform was designed within the ERC Starting Grant project Explorers 240007 (2009-2014). Explorers has been studying mechanisms allowing open-ended learning and development in robots and humans. In particular, Explorers targeted to study how the morphological properties of the body could impact the acquisition of motor or social skills. In the ERC project, we realized that if one wanted to really study the role of the body in cognition, one needed to be able to consider the body as an experimental variable: something that can be easily changed and experimented. Yet, this was so far impossible because robot platforms were developed using classical machining techniques requiring a lot of time, energy and funding. Also, classical machining techniques did not allow to build certain shapes. In Explorers, we decided to take advantage of the 3D printing revolution by transposing it to humanoid robotics, and this lead us to design the Poppy platform. All aspects of the platform were designed to be highly modular, modifiable, robust, and easily replicable in other labs for cumulative science. In just a few days, we can now systematically study how various shapes of the legs or feet influence balance in biped locomotion, or how various head morphologies will provoke different reactions when socially interacting with humans. Several research labs in Europe have already began to use the Poppy platform for their own projects (e.g. Collège de France, Bristol Robotics Lab., Inria Nancy …). Such properties of the platform then revealed a very high potential for a the much larger field of education and FabLab communities, which is the topic of this ERC PoC.

% chapter conclusion (end)
