% !TEX root = ../../thesis.tex

\chapter{Approach}

\textbf{Keywords:} rapid prototyping - body design - body as an experimental variable - experimentation - 3D printing - open-source - open science

\section{Current limitations}

In the previous chapter, we presented several works showing the importance of the robot morphology and the need to continu the research in this domain. However, we saw that none of the available commercial robot platform permits such scientific exploration.

Also, the current research practices in the Robotics field limit the diffusion and the impact of contributions. Indeed, in most cases, there is no material associated with a published paper. Meaning, only the theory is shared with the community but not the actual framework allowing to reproduce the results.

\begin{itemize}
    \item Lost of time, design a whole new robot.
    \item Morphology as an experimental variable
    \item Current robotic platforms do not permit to explore the morphology as a variable.
    \item IA haut niveau qui peut se faire en simulation avec certaines reserves.
    \item Currently, robot platform are not satisfying if we want to explore the role of morphology. They are not hackable and proto or not both easy to use and accessible.
    \item Problem with simulator VS real world
    \item Exploring control with real robot raise challenge due to error
    \item Morphologie as variable experimental
    \item Engineering on the platform not research
    \item There is two categories, commercial platform and research lab prototype.
    \item No distribution of the code
    \item no benchmark platform
    \item no hackable platform
\end{itemize}

This raises limitations for:
\begin{itemize}
    \item verifying the quality of the results presented in the paper,
    \item the reuse of the contribution as there is, in most case, a gap between the theory and the actual implementation,
    \item the collaborative work with other laboratory
\end{itemize}

However in the context of exploring control algorithm or studying the role of morphology, the actual robot is needed. Also in this context, it raises major issue concerning the way to share material associated to a scientific contribution.

In this chapter, we present a design methodology allowing to:
\begin{itemize}
    \item consider the morphology as an experimental variable.
    \item easily share our results with the scientific community
\end{itemize}



\section{Developping experimental tools}

In this context, we need to find new design methodology allowing to easily explore the role of morphology and we need to find workflow to share scientific contributions with the community. 

\subsection{The needs}
hardware platform keyword: robust, flexible, safety, breakable, repairable, stationnarity, affordable, Easy and fast to duplicate, modular, easy to set up.


\section{The body as an experimental variable}

\subsection{Rapid Prototyping}


\section{Fast and easy to duplicate} % (fold)

\subsection{Off-the-shell components} % (fold)




\section{Conclusion} % (fold)

FabLab approach: We use 3D printing, simple design, simple to use library, open source and creative commons licence.

Diffusion Open source with git repository

We find a really quick and cheap way to conceive Poppy.

Based on this work, we will be able both to explore morphology and distribute our works.

These needs share a common thread: the need of an easy to get/reproduce hardware platform.

We want to have flexible platform allowing to easily perform scientific experiments. We do not want to spend more time on debug than doing actual experiments and then we want to be able to share our research in order to have a scientific impact.



