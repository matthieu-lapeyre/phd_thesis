% !TEX root = ../../thesis.tex

% \cleardoublepage
% \newpage
% \thispagestyle{plain}
% \mbox{}

% \includepdf{/Users/matthieulapeyre/Documents/phd_thesis/media/poppy_artist.pdf}

\chapter{The Poppy diffusion} % (fold)

\cleanchapterquote{Think global, act local}{moi}

\section{Introduction} % (fold)
politecal aspect (Fablab) the economical (crowndfounding).

using the makers movement as a lever arm for the project diffusion.

\section{Makers \& Fablabs} % (fold)


http://alternatives.blog.lemonde.fr/2014/04/06/ces-projets-open-source-qui-changent-le-monde/



http://www.gizmag.com/disney-research-3d-printed-optics/24435/
http://www.3ders.org/articles/20131119-first-ever-3d-printed-electronics-set-to-launch-into-space-today.html


the ikea effect

Von Hippel, Eric (2002) "Shifting Innovation to Users via Toolkits", Management Science Vol 48, No 7, July
 Von Hippel, Eric (1986) Lead Users: A source of Novel Product Concepts.
 Management Science.
 Vol 32, No 7,
July 1986

Von Hippel, Eric (1994).
" Sticky Information and the locus of problem solving: Implications for innovation."
Management Science, 40(4) 429-439

Le concept de fablab a été inventé au MIT en 1999 avec le cours de machin "how almost build anything". The first


Souvent organisé sous forme de non-profit, Le concept est finalement de proposer un lieu d'échange ou amateurs et professionnel peuvent venir échanger et travailler ensemble autour de machine mise à disposition par l'association. Souvent une part des dépenses est pris en charge par des cotisations annuelles des memebres.

Ces lieu d'échanges et de production ont pris differentes forme appelé au choix makerspace, hackerspace ou fablab.

Since few years and the large democratization of 3D printing tools, the movement is gaining momentum and begin to be a clear and credible alternative to classic way to produce consummery good.

Indeed, it is now possible to exchange numerical data thanks to open hardware licence which are reproducible on rapid prototyping machine available in the fablab at the corner.
This new way to work and produce is so effective, the big company open intern fablab where employee can works on diferents project both for the compagny or to explore external open source project.

Une nouvelle generation de personne que l'on appelle les makers retrouvent le plaisir de mettre les mains dans le cambouit.

Ainsi some fablab project get a major impact


Le concept a donné



Poppy is an experimental tool design to be use outside the lab by

Set up an open source release and community management tools are not enough. Our work needs to face the real world.


\section{real world} % (fold)

Heidi referenced the Paradox of the Active User (pdf), which has been around as a concept since 1987. I highly recommend reading the original paper, but if you don't have time, Jakob Nielsen summarizes:
\begin{quotation}

Users never read manuals but start using the software immediately. They are motivated to get started and to get their immediate task done: they don't care about the system as such and don't want to spend time up front on getting established, set up, or going through learning packages.
The "paradox of the active user" is a paradox because users would save time in the long term by learning more about the system. But that's not how people behave in the real world, so we cannot allow engineers to build products for an idealized rational user when real humans are irrational. We must design for the way users actually behave.
\end{quotation}



\section{Production/Distribution: an alternative approach} % (fold)

\subsection{A research lab is not a Start-Up} % (fold)
Poppy includes three main parts: its mechatronic structure (skeleton and motors); its electronics; its software.

Reproducing and rebuilding the mechatronic structure is easy: the open-source skeleton can be printed on personal 3D printers (or using online services for higher quality printing), and motors are bought off-the-shelf (motors are currently not open-source, but very standard). Obtaining and using the software is very easy: just download on the Poppy web site.

But the fabrication of electronics is challenging. It is not yet possible to produce electronics components at home, and many institutional users do not have the competence or motivation to do so. There are some kickstarter projects going on a way to facilitate the process, yet they are not ready and won't be ready until a couple of years.

The current classical approach to build and distribute this electronics boards is to raise funding allowing the manufacturing of hundreds of boards which can then be sold by a distribution company. But a french research institute like Inria is not a distributor,  is not even legally allowed to do.

Furthermore, even if one could buy or build easily all components, some users (e.g. artists) might want to obtain and use an already fully assembled Poppy robot. Thus, a structure capable of building and distributing the electronics, as well as the mechatronics and/or the robot fully assembled is needed.

Yet, the mission of a research team at Inria is to do research, and find ways to apply and transfer the results of this research, but not directly to produce and sell a commercial product. If a commercial product can emerge from our research, one way to exploit it is to create a start-up company which will set up a business plan around it, probably based on a production in Asia and then a worldwide distribution to research laboratories, universities and fablabs (see Figure \ref{fig:classic})

\begin{figure}[h]
    \begin{center}
        \includegraphics[width=14cm]{classic_production_distribution.jpg}
    \end{center}
    \caption{Classical approach for technology production and distribution}
    \label{fig:classic}
\end{figure}


But Poppy is not designed to be a standard commercial product. While it might foster the creation of an economical ecosystem and jobs, its main purpose is to become an educational tool that remains open, as well as rather low cost and easily reproducible. If the goal would have been to make it profitable, it would be necessary to sell it at a much higher price. The robot wouldn't be as accessible at it should to ensure the achievement of its scientific diffusion and educational missions... We would loose the intrinsic purpose of Poppy.

\section{Toward local open factories } % (fold)

Meanwhile, the "makers revolution" is gaining momentum~\cite{anderson2012makers} and more and more Fablabs are created around the world. As a main mission of Poppy is to be a educational platform, Poppy could become a popular platform used, hacked, transformed within the natural FabLab activities. But also, and this is the direction explored below,  it would make sense that Poppy, as a whole or subsets of its components, be produced and distributed by Fablabs, and thus becoming a tool used by Fab Lab to develop and robustify the economic ecosystem in which they live.

\subsection{Toward an alternative model} % (fold)

An original and constructive organizational process would be to take advantage of the production phase for educational purposes. In this context, each fablab would have the possibility to produce, assemble and sell Poppy to local actors (see Figure \ref{fig:world_fab}). Thus the production phase can become a training support for the use of 3D printing techniques and the manufacturing of electronic circuits, and later on be exploited through selling the constructed platforms.


\begin{figure}[tb]
    \begin{center}
        \includegraphics[width=14cm]{fabusine_distribution_world.jpg}
    \end{center}
    \caption{Fabrication and distribution locally done by Fablabs}
    \label{fig:world_fab}
\end{figure}


Also in a context where fablabs need to find an economic model, several sources of income may be found thanks to the distribution of platforms such as Poppy. The first and most obvious one is the sale to local actors of fully-assembled and functional Poppy robots produced by the Fablab. But a more advanced model can emerge. Poppy is a development robotic platform: it means that it can and will be broken, meaning that Fablabs may extend theirs commercial offers. Among them we can cite:

\begin{itemize}
    \item Ensure a support (repairs, upgrades, ...) and may sell maintenance contracts with labs/school/university and even other 3rd party FabLabs.
    \item Provide a customisation service to adapt Poppy to specific needs (e.g. a university or lycée that would like to have a Poppy on wheels rather than legs)
    \item For an event or artist residency: The FabLab could rent a robot and provide a technician,
    \item Propose profesional formation to 3D printing to companies
    \item ...
\end{itemize}

From these kinds of interaction, links and collaboration between local actors and Fablabs may emerge leading to other potentially funded projects.

\subsection{Promote local collaboration} % (fold)

Beyond the act of production and sales, Poppy could become a pretext to promote the linkage and exchange between local actors from multiple backgrounds. At the scale of a city or region, we can easily imagine a distribution of roles where several FabLabs could collaborate to build and distribute different parts of Poppy depending on their motivations, skills and equipements.
Also, it helps to connect the fablabs with local actors, public/private research labs, companies, schools/universities or artists (see Figure \ref{fig:local_synergy})

\begin{figure}[tb]
    \begin{center}
        \includegraphics[height=10cm]{fabusine_local.jpg}
    \end{center}
    \caption{A synergy can emerge between Fablabs and local actors}
    \label{fig:local_synergy}
\end{figure}

\subsection{What is the role of the Flowers research team in such a process? } % (fold)

The Flowers research team's role remains essential. As the founders, designers and leaders of both the technological platform and its surrounding philosophy of openness and innovation, the Flowers team continues to improve the platform, take a central role in animating the community of users, and design new uses with scientists, educators, geeks and artists. Within this process, the Flowers team also coordinates the growing of the community of contributors and users, and designs strategies to ensure both the quality and sustainable development of the platform and its uses.

Among the tools used by the Flowers team to ensure such quality and sustainable development is through the control of the "Poppy" brand, and through policies/charters:

\begin{itemize}
\item The "Poppy" brand is owned by Inria, and the use of the brand by 3rd parties like FabLabs will only be possible through agreements ensuring that the Poppy project policies and philosophy is implemented;
\item Agreements take the form of charters/policies between Inria and FabLabs specifying guidelines to follow to ensure both quality and that each party (Inria, FabLab, users) finds its interest.
\end{itemize}

On the Inria side, the creation of an association which role would be to spin-off this technology development, community animation and quality control, is under consideration.


\subsection{conclusion} % (fold)
Poppy can be one of the first project launching this new kind of production and distribution process. The last months we met several of the main French FabLab. While their are quite enthousiast with this idea, the organisation is not completely ready to go on this way and we will certainly have to use the two ways to distribute Poppy. The first ones will be to let reseller create kit and sold them.







