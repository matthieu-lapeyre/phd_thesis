% !TEX root = ../../thesis.tex

\cleartoleftpage
\includepdf{../media/chapter_illustration/franki}


\chapter{Changing Poppy's morphology} % (fold)
\label{cha:changing_morphology}


Poppy has been designed to be a new experimental platform opening up the possibility of systematically studying the role of morphology in sensorimotor control, in human-robot interaction and in cognitive development. Indeed, as we discussed in chapter~\ref{cha:morphology-review}, a suitable design of a robot’s morphology can greatly simplify control problems, increase robustness, and open the way for new modes of interaction with the physical and social world. Thus, being able to study the body as an experimental variable, something which can be systematically changed and experimented with, is of paramount importance. Yet, until recently it was complicated because building a robot relied on heavy and costly manufacturing techniques, but 3D printing has changed the landscape of possibility.

We introduced a design methodology relying on the use off-the-shelf components and Arduino electronic architecture, for which 3D printing plays a central role in the production of mechanical parts (see chapter~\ref{cha:methodology}).

Poppy transposes this methodology to humanoid robotics, and it is now possible to explore new body shapes in just a few days. In addition, its size, weight and power actuation highly reduce the risk of self-damage if a programming error occurs, which means experimentation can be directly conducted in the real world without having to either use physical simulator or build a heavy experimental setup.

In this chapter, we present several experiments aiming to show through examples how the Poppy's morphology can be easily and quickly hacked  to explore morphological variants in the real world.
These experiments will be presented to show different aspects:

\begin{enumerate}
    \item Experimenting the role of morphology (section~\ref{sec:morphology-role})
    \item Fast exploration of morphological variants (section~\ref{sec:morphology-variable})
    % \item Adding a novel mechanism (section~\ref{sec:morphology-add-mechanism})
    \item Adding new sensors to Poppy (section~\ref{sec:morphology-adding-sensors})
\end{enumerate}



% !TEX root = ../../thesis.tex

\newpage
\section{Experimental evaluation of the role of the morphology: the thigh shape} % (fold)
\label{sec:morphology-role}

The role of morphology in robot biped locomotion has been particularly explored through the research on passive dynamic walkers~\cite{wisse2007passive}. The most famous example concerns the Tad MacGeer's work~\cite{mcgeer1990passive}. Thanks to the understanding of the intrinsic dynamics of its structure, McGeer has managed to create a 2D biped robot capable of producing several steps without any controller or motor.
The only control of this robot is obtained through the interaction between the intrinsic inertia of the structure and gravity.
This work has been pursued with the apparition of semi-passive walker combining both specific passive properties and low power actuation to increase their robustness~\cite{Anderson2005}. We can note the work of Collins~\cite{collins2005bipedal} which explored the case of semi-passive 3D biped robot. Its morphology is based on particular mass distribution, knee locking, round feet and springs on the legs to generate an efficient walking gait while keeping its lateral and frontal balance.
The concept of 3D semi-passive robot has been pushed even further with the realization of a complete humanoid robot with trunk, arms and head: the robot Denise~\cite{wisse2005three} and Flame presented in~\cite{Hobbelen2008}.

The geometry and distribution of mass in the body has complex influences on biped locomotion. Several studies have for example explored the role of the foot and ankle morphology for biped walking on both human~\cite{Adamczyk2006}~\cite{Hughes1990} and robot~\cite{hobbelen2005ankle}~\cite{Davis2010} However, to our best knowledge no research has focused on the role of the thigh for biped locomotion.  A few robots like HRP-4C~\cite{kaneko2009cybernetic} and Kenshiro humanoid~\cite{nakanishi2013design} robots seem to visually have a morphology design close to the thigh shape of Poppy, but no comparative study of the role of this shape was presented so far.

Thanks to the conception of Poppy allowing easy, cheap and fast morphology modifications, we are able to experiment the impact of various thigh shapes on the robot dynamics. In particular, in this experiment, we will focus on its bio-inspired thigh shape, bended by an angle of 6\textsuperscript{o}. We will investigate the impact of this thigh design on the balance and biped locomotion using a comparison with a more traditional straight thigh (see Fig.~\ref{fig:poppy_compared}).

% The results presented in the incoming sections have been published in the Humanoids 2013 conference proceeding~\cite{lapeyre:hal-00861110}.


\begin{figure}[!t]
\centering
    \subfloat[][bended thighs]{\label{fig:poppy_with_bended_legs}\includegraphics[width=0.35\linewidth]{poppy_bended_tigh_square.pdf}}
    \hfil
    \subfloat[][straight thighs]{\label{fig:poppy_with_classical_legs}\includegraphics[width=0.35\linewidth]{poppy_straight_tigh_square.pdf}}
    \caption{We evaluate the effect of the thigh morphology on the biped locomotion dynamic.
    Experiments are made using the Poppy humanoid platform.
    In this paper, we compare two thigh morphologies: (a) thigh bended by an angle of 6\textsuperscript{o} and (b) a more classical approach with straight thighs.}
    \label{fig:poppy_compared}
\end{figure}


\subsection{Understanding the role of the thigh shape in human} % (fold)

If we look closely at the human's femur morphology, it appears that it is inclined by an angle of 6\textsuperscript{o}. This makes the feet closer to the projection of the center of gravity (see Fig.~\ref{fig:human_thigh}) and leads to two main stability enhancements during the walking gait:

\begin{itemize}
    \item As the feet are closer to the gravity center, the necessary lateral translation of the CoG to transfer the mass of the robot from one foot to another is reduced (see Fig~\ref{fig:human_thigh}). In the case of Poppy's morphology, thanks to the $6$\textsuperscript{o} bended thigh, the lateral motion of the CoG is reduced by about 30\% ($ 5 cm$ instead of $7.1 cm$).
    \item During the stance phase, the CoG initial conditions are slighty modified. As we will explain with a simple theoritical model in the next section, the bended thigh can reduce the falling rate.
\end{itemize}


\begin{figure}
\centering
    \subfloat[][]{\label{fig:human_thigh}\includegraphics[height=8cm]{human_thigh.jpg}}
    \hfil
    \subfloat[][]{\label{fig:model_thigh}\includegraphics[height=8cm]{model_thigh.jpg}}
    \hfil
    \subfloat[][]{\label{fig:thigh_of_poppy}\includegraphics[height=6cm]{thigh_shape_large.jpg}}
    \caption{ a) Effect of the human being bended femur on the human biped locomotion.
    b) Model used for the comparison of the two thighs morphology.
    c) Actual realization of the bended thigh on the poppy platform}
    \label{fig:poppy_thigh}
\end{figure}

\subsubsection{Theoretical model} % (fold)
\label{sub:exp_theoritical_model}

We can model the situation where the robot is on one foot by an inverted pendulum with a point mass centered on the center of gravity (CoG) of the robot and the axis of rotation located at the foot position (see Fig.~\ref{fig:thigh_of_poppy}). With such model, the dynamic of the whole structure depends on:

\begin{itemize}
    \item the length $l$ of the segment extending from the foot to the center of gravity,
    \item the angle $\theta$ of the segment relative to the vertical,
    \item the gravity force $g$.
\end{itemize}

And the system follows this physical law:

\begin{equation}
    \ddot{\theta}(t) + w_0 \cdot sin(\theta(t)) = 0
\end{equation}
with:
\begin{equation}
    w_0 = \sqrt{\frac{g}{l}}
\end{equation}

\subsubsection{Intuitive expectation} % (fold)

To get a first insight of the behavior, we can linearize the system for small disturbance such as:

\begin{equation}
    \theta(t) = \theta_0 \cdot cos(w_0\cdot t)
\end{equation}
and
\begin{equation}
    \dot{\theta}(t) = -\theta_0 \cdot w_0 \cdot sin(w_0\cdot t)
\end{equation}

One can notice the position and velocity of the pendulum linearly varies with the initial condition i.e $\theta$ angle. Thus reducing this initial angle $\theta_0$ involves a direct reduction of the falling speed $\dot{\theta}(t)$ of the robot.

In the case of Poppy's geometry, the thigh bending allows a 40\% reduction of the initial angle $\theta_0$ ($\alpha = 3.8$\textsuperscript{o} against $ \beta = 6.4$\textsuperscript{o} on Fig.~\ref{fig:model_thigh}).

\subsubsection{Simulation} % (fold)

In the case of a fall, it is not possible to make the assumption of small perturbations, that is the reason why we have simulated the model in Matlab with a non-linear system and obtain the behavior represented in Fig.~\ref{fig:dynamic_thigh_model}.

\begin{figure}[thpb]
    \centering
    \includegraphics[width=0.8\linewidth]{dynamic_thigh_model.pdf}
    \caption{Comparison of the falling dynamic over time when Poppy is standing on one foot depending on  its thigh morphology: with a bended thigh of 6\textsuperscript{o} (blue) and with a straight thigh (red).}
    \label{fig:dynamic_thigh_model}
\end{figure}

If we define the center of gravity altitude as:
\begin{equation}
    z_{CoG} = l \cdot cos(\theta(t))
\end{equation}
We can express its falling speed over time as:
\begin{equation}
    \dot{z}_{CoG} = - \dot{\theta}(t) \cdot l \cdot sin(\theta(t))
\end{equation}

In this condition, if we consider the first 700ms of the system behavior simulation and compare the two systems, the mean of the CoG falling speed is reduced by around 56\% in the bended thigh case.


\subsection{Experimenting thigh properties variation with Poppy} % (fold)

The simple model described in the previous section showed that a slight inclination (6\textsuperscript{o}) of the thigh can theoretically have a significant gain regarding the lateral stability of the robot during the two main phases of the walking gait (i.e. single stance phase and double stance phase).

In this section, we describe representative experiments which evaluate the actual gain of the thigh shape on the real Poppy platform. To do this, we used both a pair of straight thighs and the bended thighs presented above. We will compare Poppy's reactions with those different legs (see Fig.~\ref{fig:poppy_compared}) on three experiments:
\begin{itemize}
    \item Evaluate the falling speed during single support stance.
    \item Measure the lateral translation to move the CoG Form one feet to the other.
    \item Record the upper body motion during biped locomotion.
\end{itemize}

\subsubsection{Single support falling velocity} % (fold)
\label{ssub:falling_velocity}
The experiment evaluates the fall velocity of Poppy when it is supported on only one foot and compare it with the theoretical results obtained in~\ref{sub:exp_theoritical_model}. To do so, the robot's head is tracked by an Optitrack\footnote{\url{http://www.naturalpoint.com/optitrack/products/v120-trio/}} device and markers are placed on the head. In postural balance on two feet, a motor order triggers the raise of a foot which unbalances the robot (see Fig.~\ref{fig:falling_experiment_dispositif}) and causes its lateral fall (see Fig.~\ref{fig:fall_of_poppy}). This experiment was repeated about fifteen times for the two cases studied, i.e. with bended legs (Fig.~\ref{fig:poppy_with_bended_legs}) and with straight legs (Fig.~\ref{fig:poppy_with_classical_legs}).

\begin{figure}[h]
\centering
    \subfloat[][]{\label{fig:falling_experiment_dispositif}\includegraphics[height=6cm]{experience_fall_illu.jpg}}
    \hfil
    \subfloat[][]{\label{fig:fall_of_poppy}\includegraphics[height=6cm]{experience_fall_mean.jpg}}
    \caption{Run of the one support stance falling experiment.
    The Poppy head has a headband with markers to track its absolute position over time.
     a) Initial perturbation done a sudden raise of one foot, b) view of the Poppy lateral fall over time.}
    \label{fig:falling_experiment}
\end{figure}

Experiments results are shown on the Fig.~\ref{fig:falling_results}. The blue color is assigned to experiments with bended thighs while the red color is assigned to straight thighs. For each case, the light color corresponds to the standard deviation and the dark color to the 95\% confidence interval of the mean value. The first figure~(\ref{fig:fall_result_position}) refers to the head altitude position over time and the second~(\ref{fig:fall_result_velocity}) to the falling velocity of the head. Dashed lines represent theoretical results obtained with the model presented in section~\ref{fig:model_thigh}. One can notice the strong similarity both on the shape and on the difference between the two morphologies studied. Yet, there is a slight time shift between theoretical and experimental results. This can be explained by the inertia of the real robot which was not took into account during the simulation.

\begin{figure}[h]
\centering
    \subfloat[][Vertical head position]{\label{fig:fall_result_position}\includegraphics[width=0.50\linewidth]{falling_compare.pdf}}
    \hfil
    \subfloat[][Vertical head falling velocity]{\label{fig:fall_result_velocity}\includegraphics[width=0.50\linewidth]{velocity_compare.pdf}}
    \caption{Results of the single support falling experiment.
    The blue color is associated with experiments conducted with bended thighs while the red color is assigned to straight thighs.
    For each case, the light color corresponds to the standard deviation and the dark color to the 95\% confidence interval of the mean value while dashed lines represent theoretical results.
    These figures shows the vertical position (a) and vertical falling velocity (b) of the head of Poppy over time for each case studied.
    The curves behavior change after 800ms is due to the fact that we catch up the robot before it touches the ground.}
    \label{fig:falling_results}
\end{figure}

These figures show a clear improvement for the Poppy version with bended thigh (blue curves) with a 200ms time shift compared to the straight thigh (as illustrated on the attached video\footnote{\url{http://flowers.inria.fr/Humanoid2013/}\label{video}}).
Thanks to this delay, the falling speed is reduced by about 56\% during the first 700ms. Thus the robot remains almost stationary for 600 ms (400ms in the case of straight thigh). The typical walking gait of Poppy is done over a period of one second so the mono-pedal stance phase last around 420 ms~\cite{lapeyre2013poppy}. Considering that the robot remains stationary during more time than the single stance phase, we can imagine that the lateral balance control will be reduced during the walking gait.



\subsubsection{Double support CoG transfer} % (fold)
\label{sub:cog_motion}


In this experiment we evaluate the necessary lateral movement of the robot to cause a displacement of its center of gravity from one foot to the other and verify the theoretical results obtained previously. For this, Poppy is placed on a force platform to measure the displacement of its center of pressure. The absolute movements of the robot are tracked with an OptiTrack device and markers placed at the head and lower back (approximately the position of the actual center of gravity). The robot is kept rigid in a neutral position and a human physically pushed it from left to right until it reaches its lateral falling limit. As this operation is not very accurate, the experiment is repeated a hundred times.

\begin{table}[h]
\centering
\begin{tabular}{|l|c|c|c|}
  \hline &      Straight tigh &                     Bended Tigh &                   diff(\%) \\
  \hline CoP & 74.6 {\scriptsize$\pm$9.0} mm &     49.8 {\scriptsize$\pm$7.7} mm & 33\\
  Head & 100.1{\scriptsize$\pm$14.4} mm&     62.9{\scriptsize$\pm$22.0} mm &  37\\
  Lower Back & 64.1{\scriptsize$\pm$11.5} mm&      43.4{\scriptsize$\pm$15.0} mm &  32 \\
  \hline
\end{tabular}
\caption{Summary of the results obtained during the experiment on the lateral motion needed to transfer the robot mass from one foot to the other.}
\label{tab:CoG_motion}
\end{table}

The table~\ref{tab:CoG_motion} presents for each area considered (i.e. center of pressure (under feet), lower back and head motion) the amplitude of the lateral motion (in millimeter) needed to translate the CoG of the robot from one foot to the other for the two versions of the Poppy thigh design. The last columns summarizes the relative difference between the two conceptions (in percent). One can note that the results show a reduction of lateral movement of around 30\%. Thanks to the shape of the thigh, the lateral displacement of the upper body required to move the CoG from one foot to the other can be reduced.


The results presented on the two first experiments show improvement for two main aspects needed during biped locomotion: lateral stability and mass transfer. In the next experiment, we will evaluate if there is a significant performance gain in a complex dynamic phase such as bipedal walking.


\subsubsection{Walking dynamic} % (fold)
\label{sub:walking_dynamic}

As explained in the introduction and description of the platform, Poppy has been especially designed to study bipedal walking and human-robot interaction.

Here the experiment consists in playing an open-loop walking pattern while the robot is guided through the physical interaction with a human. The users role is to provide both balance and control of mass transfer. By producing small lateral motion on the upper-body they can help the robot to move its CoG from one foot to another.

\begin{figure}[h]
    \centering
    \includegraphics[width=0.95\linewidth]{CPG.pdf}
    \caption{Toes trajectories generated by the walking pattern a) Kinematics of human walking
    with human's morphology b) Direct transposition of the human kinematics with Poppy's morphology
    c) Reducing amplitude of the human kinematics joints with Poppy's morphology d) Walking pattern
    used for the experiment with Poppy}
    \label{fig:CPG}
\end{figure}

The gait is based on the actual human sagittal joint kinematic~\cite{Nester2003}: hip, knee, ankle (see Fig.~\ref{fig:CPG}.a). A direct transposition of the human joint kinematic on the Poppy's morphology results in a walking speed which is too fast to be handled by users (see Fig.~\ref{fig:CPG}.b). A simple reduction of joints amplitude conducts to an unsuitable leg trajectory where toes bump into the ground during the swing phase (see Fig.~\ref{fig:CPG}.c). So to ensure enough clearance during the swing phase and suitable walking speed for the guidance with user, we modified the joints trajectories by hand to both reduce the length step and increase the foot clearance (see Fig.~\ref{fig:CPG}.d). The actual gait on Poppy is shown on the Fig.~\ref{fig:humanoids2013_cpg_on_poppy}.

\begin{figure}[h]
    \centering
    \includegraphics[width=0.9\linewidth]{walking_poppy.jpg}
    \caption{Walking gait CPG described on Fig.~\ref{fig:CPG}.d applied on the actual Poppy robot.
    The CPG generates a human-like walking gait allowing the robot to walk at 1.8km/h and involves straight leg during the stance phase.
    There is no balance control but stability is obtained through physical guidance with trained expert user.}
    \label{fig:humanoids2013_cpg_on_poppy}
\end{figure}

In this experiment we are interested by the dynamic of Poppy especially on the frontal plane and we will compare the effect of the thigh shape on this dynamic. Poppy walks on a treadmill following the walking gait described above. An expert user trained to keep the robot in the correct walking cycle provides guidance to the robot. This is done by keeping the robot in a vertical position and supporting, in a compliant manner, the lateral movement of the robot as illustrated in attached videos. The user is asked to do the best he can to minimize the movement/forces he applies in both morphologies to reduce the bias towards the two design experimented. All proprioceptive sensors are recorded at 50hz while an Optitrack device associated to markers located at the head and lower back measure the absolute displacements of the robot (voir Fig.~\ref{fig:walking_experiment}).

\begin{figure}[!h]
    \centering
    \includegraphics[width=0.8\linewidth]{dispositif_experience_marche.pdf}
    \caption{Proceeding of the walking experiment.
    Poppy is tracked by an Optitrack trio while it is walking on a treadmill set at 1.8km/h.
    An expert user provides the sagittal balance needed during all the experiment.}
    \label{fig:walking_experiment}
\end{figure}

The Poppy dynamic is recorded for around 1800 walking gait cycle for each thigh design (around 90,000 data points for each case). Data are folded over to extract the gait behavior over a gait cycle. Results are presented on Fig.~\ref{fig:walk_result}. As previously, the blue color is assigned to experiments with bended thigh, the red color is assigned to straight thigh. For each case, the light color corresponds to the standard deviation and the dark color to the 95\% confidence interval of the mean value.

\begin{figure}[!h]
    \subfloat[][Lateral head displacement]{\label{fig:head_position}\includegraphics[width=0.49\linewidth]{marche_head_motion.pdf}}
    \hfil
    \subfloat[][Lateral lower back displacement]{\label{fig:low_back_position}\includegraphics[width=0.49\linewidth]{marche_low_back_motion.pdf}}
    \hfil
    \subfloat[][Sagittal head acceleration]{\label{fig:head_acceleration_x}\includegraphics[width=0.49\linewidth]{marche_accel_x_signal.pdf}}
    \hfil
    \subfloat[][Lateral head acceleration]{\label{fig:head_acceleration_y}\includegraphics[width=0.49\linewidth]{marche_accel_y_signal.pdf}}
    \hfil
    \subfloat[][Speed of rotation in the frontal plane]{\label{fig:head_gyro_x}\includegraphics[width=0.49\linewidth]{marche_tilt_x_signal.pdf}}
    \hfil
    \subfloat[][Head inclinaison]{\label{fig:head_tilt_x}\includegraphics[width=0.49\linewidth]{marche_tilt_y_signal.pdf}}
    \caption{Results obtained during the walking experiment.
    The blue color is associated with experiments conducted with bended thighs while the red color is assigned to straight thighs.
    For each case, the light color corresponds to the standard deviation and the dark color to the 95\% confidence interval of the mean value.
    All data are folded over to extract the mean gait behavior and its standard deviation over a walking gait cycle expressed in percent.}
    \label{fig:walk_result}
\end{figure}

The two first figures (i.e. ~\ref{fig:head_position} and ~\ref{fig:low_back_position}) show the upper body lateral motion in millimeter over the gait cycle. We can notice that for the two designs, the motion pattern shown by the upper body (head and lower back) is similar. However in the case of the bended thigh (blue) the amplitude of the motion is reduced by about 45\%. Another interesting effect concerns the head perturbations shown on figures ~\ref{fig:head_acceleration_x}, and ~\ref{fig:head_acceleration_y}. Here also, patterns are similar but in the case of the bended thigh the head is clearly less perturbed by the walking dynamic with a reduction in amplitude of approximately 30\%. Five pictures have been taken while Poppy was walking and were stacked on Fig.~\ref{fig:poppy_walking_compared}. This shows a qualitative point of view of the walking dynamic for both studied case.
We can notice that the lateral motion of the version of Poppy with bended thigh~\ref{fig:poppy_walking_bended} is small compared to the version with straight thigh~\ref{fig:poppy_walking_straight}.


\begin{figure}[h]
\centering
    \subfloat[][bended thigh]{\label{fig:poppy_walking_bended}\includegraphics[width=0.4\linewidth]{marche_bended.jpg}}
    \hfil
    \subfloat[][straight thigh]{\label{fig:poppy_walking_straight}\includegraphics[width=0.4\linewidth]{marche_straight.jpg}}
    \caption{Five pictures have been taken while Poppy was walking and were stacked to obtain a qualitative view of the difference in the walking behavior in function of the morphology of the thigh.}
    \label{fig:poppy_walking_compared}
\end{figure}


\subsection{Conclusion on the thigh shape role for biped locomotion} % (fold)
We focus on the shape of the Poppy thigh and its effect on the robot dynamic. We studied the role of the morphology in the reduction and simplification of the control needed to performs complex task such as biped walking. We have presented the simple theoretical model we used for the design of Poppy thigh based on the inverse pendulum dynamic. We have conducted experiments to evaluate the improvements of the bended thigh on the real robot dynamic and compare it with the model. Thanks to the conception of Poppy allowing easy, cheap and fast morphology modifications, we were able to try another thigh design. We also use a pair of straight thighs which is a more classical approach in humanoid conception. The experimental comparison of the two thighs design confirmed the theoretical results, the bio-inspired thigh design improves Poppy dynamic on two main points useful for biped walking:
\begin{itemize}
    \item It reduces the falling velocity by almost 60\% when the robot is on one foot (single support phase).
    \item It reduces by 30\% the lateral motion needed to transfer the mass of the robot from one foot to the other (double support phase).
\end{itemize}
It is really interesting to note that such a small modification of the robot morphology has a very significant impact on the robot behavior.

These results are interesting but they do not reflect the actual Poppy dynamic while it is walking. To evaluate the effect of the bended thigh on the biped locomotion, we conducted a third experiment where Poppy is walking on a treadmill. In this experiment, we show that the bended thigh has an effect on a complex dynamic task such as the biped locomotion: it reduces the motion amplitude on the upper body of 45\% and increase the head stability of 30\%. We choose these metrics due to our experimental constraints (fixed speed, social guidance) as a qualitative evaluation of the walking gait. Moreover it provides us with an intuitive, yet incomplete evaluation of the walking. Many other measures could have been chosen or combined such as speed, energy consumption or robustness to external perturbations. It is still complicated to understand which metric is the most adapted for the robotic biped locomotion. As human being is trained to recognize biped gait, users can provide guidance to the robot for both safety of exploration and evaluation of the walking behavior.


% subsection Results (end)

% !TEX root = ../../thesis.tex

\newpage
\section{Rapid morphological exploration} % (fold)
\label{sec:morphology-variable}

In the previous section we showed how Poppy can be used to explore the actual role of morphology for humanoid behaviours. However, Poppy is a prototyping platform designed to test and experiment quickly several technological solution, especially thanks to modular properties, but until now, we did not actually evaluate it.

Therefore while we were working on a new design for Poppy's feet and exploring a design similar to "foot 1" (see \figurename~\ref{fig:pic_foot_1}), we decided to use this as a context to conduct an experiment into multiple variations of the foot morphology as an illustration of the methodology we have initiated with Poppy and presented in chapter~\ref{cha:methodology-review}.


The aim of this experiment is to quickly explore the effect of foot morphology on stability. Here, we are particularly interested in the stability of the head after a stepping impact. These impacts are quite challenging to simulate realistically and the natural compliance of the Poppy platform means it is even more important to test this on the real robot.

For the sake of lightness, the initial design of Poppy's feet only had one degree of freedom (DoF): pitch rotation. This configuration carried the inconvenience of preventing a proper parallel foot/ground contact. Thus, we developed several different feet with two degrees of freedom. Along with a standard motorized 2 DoF flat foot design, we also wanted to explore passive joints with springs. The use of passive joints allows for both lightness and reactive torque for stability.


\begin{figure}[p]
\centering
    \subfloat[][Foot\_1]{\label{fig:pic_foot_1}\includegraphics[width=0.45\linewidth]{pic_foot_1.JPG}}
    \hfil
    \subfloat[][Foot\_2]{\label{fig:pic_foot_2}\includegraphics[width=0.45\linewidth]{pic_foot_2.JPG}}\\
    \subfloat[][Foot\_3]{\label{fig:pic_foot_3}\includegraphics[width=0.45\linewidth]{pic_foot_3.JPG}}
    \hfil
    \subfloat[][Foot\_4]{\label{fig:pic_foot_4}\includegraphics[width=0.45\linewidth]{pic_foot_4.JPG}}


    \subfloat[][Blueprints of the various foot designs studied in this experiments.]{\label{fig:foot_experience_blueprint}\includegraphics[width=0.95\linewidth]{foot_experience.pdf}}
    % \subfloat[][]{\label{fig:pic_shoe}\includegraphics[width=0.48\linewidth]{pic_shoe.JPG}}
    \caption{Visual and technical descriptions of the foot designs explored in this experiments.}
    \label{fig:foot_variants}
\end{figure}


\begin{table}
    \begin{center}
        \begin{tabular}{ l c c c c }
        \hline
        \textbf{Type} & \textbf{Foot 1} & \textbf{Foot 2} & \textbf{Foot 3} & \textbf{Foot 4}\\
        \textbf{Double rotation} & Passive & No & Active & Passive\\
        \textbf{Human-like foot} & Yes & Yes & No & Yes\\
        \textbf{Toes} & Yes & No & No & Yes\\
        \textbf{Rotation axis height} & 75.70 mm & 33 mm & 39 mm & 35.5 mm\\
        \hline

        \end{tabular}
        \caption{Table summarizing the different types of feet used.\\
        \textbf{Passive double-rotation:}  one active rotation (motor: Dynamixel MX 28) for the sagittal plan and a passive rotation for the frontal plan with two springs.\\
        \textbf{Active double-rotation:}  A two motorized rotations (sagittal plan and frontal plan). No double-rotation:  one motorized rotation (sagittal plan).\\
        \textbf{Human-like foot:} a foot design resembling a human foot of a two year-old child (size: 130.7 mm shoes size: 23 EU). The feet were tested with and without shoes.\\
        \textbf{Rotation axis height:} the height between the axis of rotation of the sagittal plan and the floor without shoes.\\
        \textbf{Toes:} Indicates that the foot has toes.
        }
        \label{tab:table_feet}
    \end{center}
\end{table}


Moreover, it appeared that a proper foot/ground contact with convenient friction was difficult to obtain based only on 3D-printed material. One simple solution to this problem is to use a shoe which can provide a high friction and adapt slightly to imperfections on the ground. Furthermore, this solution also allows keeping the feet close to humans ones. Thus, the feet tested (with the exception of the flat foot) were designed from a moulding of the interior of a shoe. It is to be noted that we also included passive toes (with springs) on some of the feet tested for future work on locomotion. These toes should not have any significant impact on the criterion tested.


\subsection{Experimental setup} % (fold)
\label{sub:experimental_setup}

For this experiment, the robot simply stands upright secured by a slack strap on a fixed gantry. Different markers on the robot are tracked by a motion capture system at 100Hz (Natural Point OptiTrack). See figure \ref{fig:setup} for more details.

\begin{figure}[ht]
    \begin{center}
        \includegraphics[width=0.9\linewidth]{pic_setup.jpg}
    \end{center}
    \caption{Experimental setup. The robot is secured by slack strap on a gantry and tracked by an OptiTrack trio device. Markers are placed on the feet, hips, abdomen, and on the head}
    \label{fig:setup}
\end{figure}

Four different feet were tested (cf. Table~\ref{tab:table_feet}). Three out of the four feet were tested both with and without shoes.


\subsection{Experiments} % (fold)

The feet were tested with a very simple discrete movement (see \codename~\ref{code:foot_mouvement}), representative of the kind of impacts that occur during walking. The robot performs a single step leftward with the left leg. The left foot is lifted (3cm) and then put back on the ground with a slight lateral displacement towards the exterior (5° at the level of the hip). The duration  of the whole movement is about 0.4 s and repeated 20 times for each configuration.

\lstinputlisting[
    language = Python,
    caption = {Discrete mouvement executed on Poppy},
    label = {code:foot_mouvement},
    float = p]
    {code/foot_mouvement.py}


\subsection{Results} % (fold)

Figures \ref{fig:head_x}, \ref{fig:head_y} and \ref{fig:head_z} respectively show the evolution of the position of the head marker in the $x$, $y$ and $z$ axis for each foot tested. Dotted vertical lines indicate the beginning and the end of the leg movement.

These figures show that the dynamics of the robot are not trivial, even for the simple movement we tested, the standard deviation is not negligible and shows how chaotic the reaction of such an impact can be. This particularity is another proof of the significance of the use of experimentation versus simulation.

We can clearly see that foot 3 (standard flat foot) behaves quite differently than the other feet tested. In particular in the $x$ and $y$ directions, we see that with this foot the head tends to move more towards the exterior (left of the robot) and towards the rear.
%% These differences may be explained by the larger ground contact surface
%% provided by the flat feet.

Regarding the effect of the shoes, results are less clear but most of the time (except for foot 1) differences occur between a given foot with and without shoes. The friction with the ground can explain these differences. Bare feet tend to slip more than those with shoes.


\begin{figure}[ht]
    \begin{center}
        \includegraphics[width=\linewidth]{head_x.pdf}
    \end{center}
    \caption{Evolution of the position of the head in the $x$ axis for
    each foot tested (see \figurename~\ref{fig:foot_variants} for illustration of each foot)}
    \label{fig:head_x}
\end{figure}

\begin{figure}[ht]
    \begin{center}
        \includegraphics[width=0.95\linewidth]{head_y.pdf}
    \end{center}
    \caption{Evolution of the position of the head in the $y$ axis for
    each foot tested (see \figurename~\ref{fig:foot_variants} for illustration of each foot).}
    \label{fig:head_y}
\end{figure}

\begin{figure}[ht]
    \begin{center}
        \includegraphics[width=0.95\linewidth]{head_z.pdf}
    \end{center}
    \caption{Evolution of the position of the head in the $z$ axis for
    each foot tested (see \figurename~\ref{fig:foot_variants} for illustration of each foot).}
    \label{fig:head_z}
\end{figure}


This first experiment allowed us to determine that the use of an active double rotation of the ankle may not be mandatory. Indeed, the behaviors observed with the passive feet were even better than with the flat feet with active rotation. Although a clear interpretation of this phenomenon is still difficult to propose, some hypotheses related to the weight (with one more motor feet are heavier) and the area of surface in contact with the ground (flat foot surface is bigger) have to be investigated.

Moreover, we observed that the shoes added extra friction in relation to the ground without really impairing the stability. Although rarely used in humanoid robotics, these early results encourage us to explore this possibility in more depth.

Finally the most important aspect for us was to actually evaluate the amount of time needed to conduct such experiments with Poppy. The starting point was "foot 1" as it was the work in progress. Thus morphological design modifications only concern foot 3 and 4:
\begin{itemize}
    \item \textbf{Foot 3 (flat):} Modifying Poppy’s initial foot design to permit the integration of two Dynamixel motors and the associated flat feet required 16 hours of CAD design. The printing of the whole required part (2 legs, 2 feet and 2 ankles) took approximately 30 hours on a low-cost FDM printer (Makerbot Replicator 2).
    \item \textbf{Foot 4:} While the difference with foot 1 concerns only one parameter (i.e. the joint position), the modification needed to produce foot 4 based on foot 1 was done in approximately 2 hours of CAD. Then the printing of the new part was achieved in 10 hours.
\end{itemize}

Then, conducting the whole experiment (i.e. design the leg motion, establishment of the experimental setup and data acquisition) was achieved in about one week with two people. \textbf{In particular, the actual experimentation involving changing Poppy's feet seven times and acquiring at least 20 trials for each took less than two days.}

\subsection{Reuse of this experiment} % (fold)

Everything necessary to obtain and use Poppy is available on our GitHub project page: \url{www.github.com/poppy_project}. Also, to complete the illustration of this Poppy use-case, we diffuse along with the present paper:
\begin{itemize}
     \item the whole setup materials i.e. the code used for the experiment and the 3D files to reproduce/modify each foot,
     \item the raw data acquired that include for each trial: all markers position, head IMU measurement and the complete motors data (proprioceptive position evaluation overtime),
     \item the code used to extract and plot the results presented.
\end{itemize}

All these materials are available on the repository associated with this experiment: \url{https://github.com/matthieu-lapeyre/Humanoids2014} and can be freely used e.g. for further investigation with the acquired data, or to reproduce and extend the experiment.


% % !TEX root = ../../thesis.tex
\newpage
\section{Add novel mechanism on Poppy} % (fold)
\label{sec:morphology-add-mechanism}

\textbf{context:} Poppy has been made to allow quick and cheap explore and experiment of morphology variation (i.e. change its hardware).
\textbf{need:} We are interested in the design of more under-actuated robots. Especially we want to explore semi-passive ability and use natural body properties rather than actuation power to achieve dynamic tasks. Until now, the humanoid biped locomotion has mostly been achieved using ZMP control leading to walking gait with the knees alway bended. The permanent high torque and foot impacts applied to the knee requires high-power actuator.
\textbf{Task:} With Poppy, we wanted to explore mechanical design permitting to reduce the required power. We decided to test the use of a semi-passive knee joint based on a MX-28 Dynamixel motor completed by a parallel spring system.
\textbf{objet:} We will explain how we create this mechanism by changing Poppy's leg mechanical design and we will show the results we got during walking experiments.

We explore the design of a semi-passive mechanism aiming to assist the motor during two critical phase of the walking gait:
\begin{enumerate}
    \item the foot impact which can produces abrupt raise of load in the knee during stance phase,
    \item the flexion of the leg during swing phase.
\end{enumerate}

\subsection{Semi-passive knee mechanism principle} % (fold)
For this purpose we chose a mechanism inspired by the one of Gini and Scarfogliero~\cite{gini2009new} involving two traction springs positioned in parallel with the knee joint in such a way that there is two low-potential solutions, one when the leg is straight and one when the leg is bended (see \figurename~\ref{fig:Gini_knee}).

\begin{figure}[]
\centering
    \subfloat[][Actual design of the robot knee.]{\label{fig:}\includegraphics[height=5cm]{Gini_knee_design.jpg}}
    \hfil
    \subfloat[][Mechanism principle.]{\label{fig:}\includegraphics[height=5cm]{Gini_knee_mechanism.jpg}}
    \caption{Gini and Scarfogliero designed a bio-inspired knee joint which uses parallel traction springs for both bend the leg and keep it straight. Illustrations extracted from~\cite{gini2009new}.}
    \label{fig:Gini_knee}
\end{figure}

Therefore this mechanism can participate in the leg dynamic and assist the motor during two walking main phases:
\begin{itemize}
    \item They help to keep the leg straight during the support phase without any motor control.
    \item During the swing phase, they participate to the flexion of the leg.
\end{itemize}

These two modes can be passively switched by the actual knee angle, yet we have to determine which angle is the most suitable. Considering the human knee kinematic (see Fig.~\ref{fig:human_knee_kinematic}), we chose to change mode at $\theta_{knee} = 20+5$\textsuperscript{o}  which corresponds to a transition between the preparing stance phase and the swing phase.

\begin{figure}[thpb]
    \centering
    \includegraphics[width=0.6\linewidth]{knee_kinematic.pdf}
    \caption{Actual human knee flexion kinematic during the walking gait~\cite{Nester2003}. We can identify two main phases corresponding to the preparation of the stance phase and the swing phase. The main difference is the amplitude of the motion i.e $<20$\textsuperscript{o}  for the stance phase and $>20$\textsuperscript{o} for the swing phase.}
    \label{fig:human_knee_kinematic}
\end{figure}

\subsection{Semi-passive knee design} % (fold)

We performed a parametric optimization both on the position of the spring ties ($M_T$ and $M_L$) and on its characteristic ($K$, $L_0$, $D_i$, $F_{max}$, $L_{max}$) (see Fig.~\ref{fig:knee_conception}) to try to match the above mentioned criterion. These criterion are modeled as condition on the resultant torque:

\begin{itemize}
    \item $C(\theta=0) < -0.4$: Locking of the knee, where $0.4 N.m$ is the necessary torque to keep the leg straight.
    \item $C(\theta=25 \textsuperscript{o} ) = 0$: Transition between the two behaviors
    \item $C > 0$ if $\theta > 25 deg$: Helps the motor to lift the leg.
    \item $ max(\abs{C(\theta)}) < \frac{C_{MX-28}}{2}$: The actuator $MX-28$ should always be powerful enough to control the joint motion.
\end{itemize}

\begin{figure}[h]
    \centering
    \includegraphics[width=0.9\linewidth]{knee_spring.jpg}
    \caption{Spring parameters to optimized}
    \label{fig:knee_conception}
\end{figure}


The resultant torque $C$ generated by springs in function of the knee flexion $\theta$  ($n_{spring} = 2$) is computed as follow:

\begin{equation}
    C(\theta) = n_{spring} \cdot \overrightarrow{OM_L}{\rvert}_{R_{thigh}} \wedge \overrightarrow{F}(\theta) \cdot \overrightarrow{z}
\end{equation}

with:

\begin{equation}
    \norm{ F(\theta)} = K \cdot \left ( L(\theta) - L_0\right )
\end{equation}

\begin{equation}
    L(\theta) = \quad \norm{ \overrightarrow{M_T M_L}{\rvert}_{R_{thigh}}}\qquad
\end{equation}

\begin{equation}
    \overrightarrow{OM_L}{\rvert}_{R_{thigh}}  = \overrightarrow{OM_L}{\rvert}_{R_{leg}}  \cdot
    \begin{bmatrix}
        cos(\theta) & -sin(\theta) & 0 \\
        sin(\theta) & cos(\theta) & 0 \\
        0 & 0 & 1\\
    \end{bmatrix}
\end{equation}



We use an iterative selection on these criteria to determine the appropriate characteristics for the spring.

\subsubsection{Minimizing stresses on the structure} % (fold)
\label{par:minimize_stresses_on_the_structure}

The length of the lever arm is constrained by the dimensions of the legs, resulting in an increase of the force generated by the spring to produce the desired torque on the knee.

By maximizing the following criterion with the constraint $C(\theta=25 \textsuperscript{o} ) = 0$:
\begin{equation}
     c_1 = \frac{C_{max}}{F_{max}^2}
\end{equation}

We were able to determine the ties specific location ($M_T$ and $M_L$), for both minimizing mechanical stress and changing the torque direction for $\theta = 25\textsuperscript{o}$,


\begin{equation}
    M_T={\left \{2,39,0 \right \}_R}_{thigh}
    \qquad
    M_L = {\left \{-12,23,0 \right \}_R}_{leg}
\end{equation}

and constraints concerning the springs characteristics:
\begin{equation}
    L_{min} < 42.6 mm
    \qquad
    L_{max} > 65.12 mm
\end{equation}


\subsubsection{Ties strength} % (fold)
\label{par:ties_strength}

We calculated the minimum diameter of the ties so that it can withstand the constraints imposed by the spring with a beam theory model:
\begin{equation}
    D_{min}= \sqrt[3]{ \frac{32 \times  C_s \times F_{max} \times l_{tie}}{2 \pi \times \sigma_{MaxPolyamide}} }
\end{equation}
By considering Poppy's parameters and a coefficient of safety $C_s = 5$, we have found that the spring must respect the criterion $D_{min} > 6.5 mm$.

\subsubsection{Obtained behavior} % (fold)
\label{sub:obtained_behavior}

Considering the desired spring behavior and geometrical conditions, an automatic selection over 720 different springs\footnote{pre-selection of springs in the vanel.com catalogue} was performed. Only 5 springs satisfied all criteria. For the Poppy platform we chose a spring with the following characteristics: $\{ D_i=9.6mm$, $L_0=42mm$, $K=1620N.m^{-1}$, $F_{max}=81.7 N$, $L_{max}=72.8 mm \}$ inducing a resultant behavior shown in Fig.~\ref{fig:knee_feature}. As we can see, even if the torque applied by the spring is quite low ($C_{max} = 0.74 N.m$), the force subjected to spring ties is up to $40N$. The shape of this ties has been optimized using FEA (Finite Element Analysis) in order to handle the stress.

\begin{figure}[h]
    \centering
    \includegraphics[width=0.95\linewidth]{torque_knee.png}
    \caption{Theoretical semi-passive mechanism behavior. The blue line corresponds to the torque applied by the springs on the leg according to the flexion angle of the knee. The red line corresponds to the force that the springs applied on ties.}
    \label{fig:knee_feature}
\end{figure}

\subsection{Experimentation on Poppy} % (fold)

An illustration of the real behavior is shown in the videos\url{https://vimeo.com/63839782}

\begin{figure}[h]
    \begin{center}
        \includegraphics[width=0.9\linewidth]{poppy_semi_passive_knee.jpg}
    \end{center}
    \caption{Caption here}
    \label{fig:figure1}
\end{figure}


\begin{figure}[!h]
\centering
    \subfloat[][without springs]{\label{fig:knee_wout_spring}\includegraphics[height=7cm]{knee_wout_spring.png}}
    \hfil
    \subfloat[][with traction spring]{\label{fig:knee_w_spring}\includegraphics[height=7cm]{knee_w_spring}}
    \caption{Motors fully compliant \url{https://vimeo.com/63839782}}
    \label{fig:}
\end{figure}




% !TEX root = ../../thesis.tex

\newpage
\section{Extending the sensor apparatus of Poppy} % (fold)
\label{sec:morphology-adding-mechanism}


Poppy has been designed following a methodology (presented in section~\ref{REF}) which makes easy the hacking of the platform. With its 3D printed structure, it is quite easy and straightforward to modify its mechanical parts, unfortunately we cannot (yet) print complex electronics circuits and components. We therefore chose to fork the Arduino Due board and design a custom I/O board (detailed in section~\ref{REF}). As its name suggests, this board has for main purpose to ensure the several inputs/outputs of the robot and offers:
\begin{itemize}
    \item 2 Dynamixel buses (TTL),
    \item 2 internal USB and 2 external USB ports,
    \item analog and digital pins available on a classic Arduino Due which can be use as direct input/output or for communication buses such as UART, I2C or SPI.
\end{itemize}
Thus there are many more I/Os than required for Poppy. These extra ports has been intended to let Poppy users extend its sensorimotor space and adapt it to their needs.


During our first trials to design walking a primitive with Poppy, we have been interested in the measurement of under feet pressures but the simple foot design Poppy has, does not involve such sensors. With a traditional robotic platform, we should have to either use the available sensors, here the load measurement in the ankle Dynamixel motor, or add an external device with its own power supply and communication system.

With the Poppy electronic modularity, we can hack the robot and integrate new sensors. Then they can be plugged on the I/O board for communication and power supply needs.

To provide an example of how we can actually hack the Poppy robot, we will explain here what we did to integrate force sensors under the feet and acquire the data with the pypot library.

\subsection{Principles} % (fold)
To obtain measurement of the pressure variation under our Poppy's feet we used FSR sensors from Interlink Electronics (see \figurename~\ref{fig:FSR_explode_view}). The FSR sensor will vary its resistance depending on how much pressure is being applied to the sensing area. The harder the force, the lower the resistance is. These sensors are low-cost -6\$ each- yet theirs behaviors are very non-linear (see \figurename~\ref{fig:foot_sensor_behavior}) and the calibration is quite variable depending on the production batch and the thermal conditions. So we cannot expect having precise results.

\begin{figure}[h]
\centering
    \subfloat[][Exploded view\footnote{Illustration credit: \url{http://www.openmusiclabs.com/learning/sensors/fsr/}} of a FSR sensor]{\label{fig:FSR_explode_view}\includegraphics[height=6.3cm]{FSR_explode_view.jpg}}
    \hfil
    \subfloat[][Measured FSR sensor resistance depending on the applied force.]{\label{fig:FSR_behavior}\includegraphics[height=6.3cm]{FSR_behavior.pdf}}
    \caption{The FSR force sensors are cheap but they have a really non-linear behavior and are not very precise.}
    \label{fig:}
\end{figure}

The acquisition of the resistance can be done indirectly by designing a voltage-divider\footnote{A voltage divider is a linear circuit that produces an output voltage $V_{out}$ that is a fraction of its input voltage $V_{in}$. It often consists of 2 resistors in series.} and using the FSR resistance variations to make the voltage output varies (see \equationname~\ref{eq:voltage-divider}). This voltage ($V_{out}$) can be then measured by an analog input of an Arduino board (see \figurename~\ref{fig:foot_sensors_test}).

\begin{equation}
    V_\mathrm{out} = \frac{R}{R+R_\mathrm{FSR}} \cdot V_\mathrm{in}
\label{eq:voltage-divider}
\end{equation}


\begin{figure}[!h]
\centering
    \subfloat[][A FSR sensor connected with Arduino nano board.]{\label{fig:foot_sensors_test}\includegraphics[height=5cm]{force_sensors_test.jpg}}
    \hfil
    \subfloat[][Simple testing assembly with 4 voltage dividers with FSR sensors and potentiometers plugged on an Arduino nano board]{\label{fig:}\includegraphics[height=5cm]{foot_sensors_proto.jpg}}
    \caption{We can easily measure the resistance variation of a FSR sensor using a voltage divider with the $V_{out}$ connected to an Arduino analog port.}
    \label{fig:test_sensors}
\end{figure}

\subsection{Design of the voltage divider to reduce the sensor non-linearity} % (fold)

% subsection design_of_the_voltage_divider_to_reduce_the_non_linearity (end)
A well tuned voltage-divider can help to reduce the non-linearity of the FSR sensors. Thus we conducted an optimization on the constant resistor choice depending on:
\begin{itemize}
    \item the Arduino analog precision: $1024$ values for a $5V$ input range,
    \item the use of the Dynamixel tension as voltage input i.e. $14V$,
    \item the standard resistor E12 precision series,
\end{itemize}

With an objective function sets to minimize the difference between the actual voltage-divider behavior and the perfect linear behavior $V_\mathrm{out}(F) = \alpha \cdot F$ with $V_\mathrm{out}(3.5kg) = 5V$ (see red curve on \figurename~\ref{fig:obtained_FSR_behavior}), we obtained the choice of a $180\Omega$ resistor for the constant resistance of the voltage divider (see \figurename~\ref{fig:FSR_best_resistor}). The best behavior found is plotted in blue on the \figurename~\ref{fig:obtained_FSR_behavior}.

\begin{figure}[h]
\centering
    \subfloat[][Optimization of the voltage divider constant resistor: distance between the output behavior with respect to a linear behavior.]{\label{fig:FSR_best_resistor}\includegraphics[width=0.48\linewidth]{criteria_dist.pdf}}
    \hfil
    \subfloat[][Theoretical behavior (blue) of the output voltage with respect to the applied force with a $14v$ input and a $180\Omega$ compared with the objective linear behavior.]{\label{fig:obtained_FSR_behavior}\includegraphics[width=0.48\linewidth]{voltage_behavior.pdf}}
    \caption{The design of the voltage divider for each FSR sensor is done by optimizing the output behavior toward a ideal linear behavior.}
    \label{fig:foot_sensor_behavior}
\end{figure}



\subsection{Integration of foot pressure sensors on Poppy} % (fold)

When we did the integration of foot sensors on the Poppy, its feet were still a really simple and flexible 3D printed part. The actual force transmission was done by the shoes. We therefore had to directly attach the sensors below the shoes. Because Arduino nano board has 8 analog inputs, we have added 8 sensors under each foot (see \figurename~\ref{fig:poppy_foot_sensors}) but actually only used 5 (the big ones) and integrated the arduino nano in the leg. While it was a hack of a real shoe and not just a print of new part, the intervention was quite annoying but still achievable in one day. Here we have chosen to use USB cable to plug each Arduino nano in the Poppy's head but it could also have be done using UART, SPI or I2C communication.

\begin{figure}[h]
\centering
    \subfloat[][]{\label{fig:poppy_foot_sensors}\includegraphics[height=7cm]{foot_sensors.jpg}}
    \hfil
    \subfloat[][]{\label{fig:poppy_nano_integration}\includegraphics[height=7cm]{poppy_leg_arduino_nano.jpg}}
    \caption{}
    \label{fig:poppy_foot_sensors}
\end{figure}


As we explained in section~\ref{REF}, the Arduino programming language bring the low level programming accessible to anyone. The \codename~\ref{code:arduino_foot_sensor} shows that we actually uploaded on each Arduino nano board. With just 10 lines of code we can stream the values of 5 pressure sensors.

\lstinputlisting[
    language = C++,
    caption = {Arduino code to read force sensors data},
    label = {code:arduino_foot_sensor},
    float,
    floatplacement = H]
    {code/foot_force_sensors.ino}

Then we just have to create a novel sensor controller in pypot (see section~\ref{REF} for details) which describes the I/O communication and get the desired values (see \codename~\ref{code:pypot_foot_sensor}). Here again, the design of the pypot library makes this task easy, only 20 lines of code are required to get access to add a novel sensor and create variable to obtain its value.

\lstinputlisting[
    language = Python,
    caption = {Example of Python code written to add custom foot sensors in pypot. The \emph{FootIO} class describes how we can read the data from the Arduino nano placed in the foot. The \emph{FootPressure} class is the sensor controller which is called by the pypot to synchronize the sensorimotor space of Poppy.},
    label = {code:pypot_foot_sensor},
    float,
    floatplacement = H]
    {code/foot_io.py}


\subsection{Measured data} % (fold)
With our novel sensors, we conducted similar walking experiment as the one explains in section~\ref{REF} and shows on the \figurename~\ref{fig:humanoids2013_cpg_on_poppy} and recorded at $50hz$ the measured force variations under Poppy's feet.

The sensors are not very precise but as we can see on \figurename~\ref{fig:poppy_GRF}, the variation of the ground reaction force (mean of the 5 force sensors) over the gait cycle has a similar M-shape as the one we can find in human gait (see \figurename~\ref{fig:human_GRF}). Also we can notice than the reaction is slightly different between the two foot (\figurename~\ref{fig:right_GRF} Vs \figurename~\ref{fig:left_GRF}).

\begin{figure}[!h]
\centering
    \subfloat[][]{\label{fig:right_GRF}\includegraphics[width=0.48\linewidth]{right_GRF.pdf}}
    \hfil
    \subfloat[][]{\label{fig:left_GRF}\includegraphics[width=0.48\linewidth]{left_GRF.pdf}}
    \caption{}
    \label{fig:poppy_GRF}
\end{figure}


\begin{figure}[!h]
\centering
    \subfloat[][]{\label{fig:}\includegraphics[width=0.6\linewidth]{human_GRF.jpg}}\newline
    \subfloat[][]{\label{fig:}\includegraphics[width=\linewidth]{human_GRF_variation.jpg}}
    \caption{}
    \label{fig:human_GRF}
\end{figure}


The bad precision of the sensors prevents us from affirming conclusions but it can still give insights to understand the walking behavior of Poppy:
\begin{enumerate}
    \item The second peak of the M-shape corresponding to the toe impulsion is weak or inexistent on Poppy. Indeed, when we look at the video of the walking gait made by Poppy, we can notice it barely uses its toes.
    \item The first peak of the M-shape is very strong. Either the walking gait of Poppy was fast or the structure is too rigid and does not absorb correctly the initial impact.
\end{enumerate}

We can therefore explore some improvements axes:
\begin{enumerate}
    \item Explore why the current walking behavior does not involve clearly the passive toes. Is it cause of the walking primitive design or the mechanical design of the toes ?
    \item The initial impact is not desirable toward the achievement of a self-balanced walking behavior. We should explore solutions to absorb it.
\end{enumerate}










\section{Discussion \& Conclusion } % (fold)

The work presented in this chapter summarizes several experiments conducted during this thesis.
Unlike the work presented in chapter~\ref{cha:morphology-review}, our experimental results can be considered as preliminary and incomplete to clearly show and demonstrate the impact of the robot morphology over its dynamic and control.

Indeed, in this thesis, we are more interested in finding an appropriate methodology to explore easily in the real world the robot morphology impact rather than the actual exploration of the morphology. In this context, experiments conducted are more for illustration and evaluation of the actual way to change and hack Poppy. Toward this goal, confronting our methodology to real usage has been really instructive both to validate the design methods and highlight some non-optimized point.

After these experiments, we focused ourselves on improving the platform to make it more easy-to-use, easy-to-hack and robust to the real world. In this way, we decided to go outside the laboratory and put Poppy in non-expert users' hands. Indeed, as we explained, we are trying to construct a multidisciplinary community involving, therefore, multiple profiles more or fewer experts in robotics. Among these profiles, two potential usages draw particularly our attention: the education and the art, and will be discussed in the next chapter.


