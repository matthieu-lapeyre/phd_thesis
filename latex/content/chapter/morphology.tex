% !TEX root = ../../thesis.tex

\chapter{Application: experimenting the role of morphology with Poppy} % (fold)
\label{cha:exploring_the_role_of_morphology}


In the related works we presented some scientific and artistic research showing the emergence of complex behavior thanks to adapted morphologies. In this thesis, we suggest a methodology to easily explore variation of morphology in the context of biped locomotion.
For this purpose we designed Poppy which permits, thanks to our design process, to quickly change its morphology.


\emph{We used our intuition and large amount of experience to tune the robot’s controllers, and make design improvements, until it walked. Even after obtaining a successful walking motion, we did not manage to create a simulation that walked successfully using the same controller parameters. We tried very hard with some of the best people, but we didn’t succeed. The reason was, I think, that our type of control (using the emergent behavior of a set of simple reflex-like controllers) was highly sensitive to hardware effects like friction. Normally, one uses a local joint controller to make the joint follow a desired trajectory independent of the exact amount of friction. The local controller “abstracts these hardware effects away”, if you know what I mean. This makes the behavior of the whole system quite predictable. However, in our robots, we did not have this kind of abstraction as we were not following trajectories, and thus a little bit of extra friction has an effect on the entire motion.
So, the simulations that you find in our papers were the only ones that we did. As I said above, we did spend a long time making a high-fidelity model in Adams, and also using other methods, but eventually we gave up without success.}
\textbf{Martijn Wisse (Delft - Denise/Flame)}


To explore the role of morphology we need to do experiments on a real platform in the real world. The morphology became a experimental variable which has to be tuned, meaning you need to change it.
We decided to conduct experiments as applications of the way to explore the role of morphology with Poppy in the case of biped locomotion.

In this chapter,  we suggest to explore the impact of the thigh shape on the lateral stability and evaluate the efficient of foot designs for balance and biped locomotion.

\section{Evaluation of the role of a bio-inspired thigh shape} % (fold)
\label{sec:evaluation-thigh}

\subsection{Effect of the bended thigh} % (fold)
\label{sub:effect_of_the_bended_thigh}

% subsection effect_of_the_bended_thigh (end)

\subsection{Experiments} % (fold)
\label{sub:experiments}

% subsection experiments (end)
\subsection{results} % (fold)
\label{sub:results}

MESSAGE!
% subsection Results (end)


\section{Exploring foot and ankle shape for biped locomotion} % (fold)
\label{sec:exploring_foot_and_ankle_shape_for_biped_locomotion}

% section exploring_foot_and_ankle_shape_for_biped_locomotion (end)

\section{Conclusion} % (fold)
\label{sec:conclusion}

% section conclusion (end)

\section{Discussion} % (fold)
\label{sec:discussion}

% section discussion (end)

Finding - What
Experiments conducted showed that the thigh shape has a strong impact on the lateral biped stability by improving it by 40 to 50%.

Conclusion - So What
Little change in the robot morphology (a thigh bended by 6°) can lead to major change in the dynamic behavior.


% chapter exploring_the_role_of_morphology (end)
