% !TEX root = ../../thesis.tex

% \cleartoleftpage
% \includepdf{../media/chapter_illustration/jansen.pdf}

\chapter{Exploring robotics morphology: some fascinating work} % (fold)
\label{cha:morphology-review}

\cleanchapterquote{The cognition needs a body to think}{Rodney Brooks}

\section{Introduction} % (fold)

\begin{figure}[!b]
\centering
    \subfloat[][The turtle robot.]{\label{fig:walter_robot}\includegraphics[height=6cm]{walter_turtoise_robot.jpg}}
    \hfil
    \subfloat[][Demonstration of obstacle avoidance behaviour.]{\label{fig:turtle_behavior}\includegraphics[height=6cm]{turtoise_behavior.jpg}}
    \caption{The W. Grey Walter's turtle was a really simple robot using direct analog connexion between light sensors and wheel actuator. The way sensors and actuators were plugged determined the robot behaviour. It could demonstrate complex behaviour such as obstacle avoidance or returning to its recharging station.}
    \label{fig:turtle_robot}
\end{figure}


In 1949, Elmer and Elsie, also known as turtle robots (see \figurename~\ref{fig:walter_robot}), created by the cybernetic pioneer W. Grey Walter, could be considered as one of the first robots in the robotics modern history era (1950-now). Back at this time, the transistor was just invented (1948)~\cite{brinkman1997history} and calculus was done with mechanical machines (see the focusbox). The turtle robot was entirely analogical but was able to demonstrate complex behaviours (see \figurename~\ref{fig:turtle_behavior}). Without any "reflexion" or internal representation of itself and the world, this robot, thanks to its conception and the direct analogical interaction between sensors and actuators was able to avoid obstacles and reach its charging station~\cite{walter1950imitation}.
These complex behaviours, which can be compared with the ones found in nature, were in fact done without any kind of intelligence and were actually emerging from the interaction between the robot morphology (i.e. where sensors are placed and how they are connected with actuators) and the robot environment (i.e. light sources).


\subsection{The cognitivist approach limits} % (fold)


With the arrival of numeric computers, researchers imagined the opening of a field where it could be possible to replace pre-wired analogical electronic behaviours by the use of computer running programs. Not dependent on the hardware platform, robots would be therefore more versatile.
The artificial intelligence (AI) term was introduced in a workshop organized in 1956 by a MIT professor John McCarthy (REF). Globally participant were convinced, that by using the notion of computation or abstract symbol manipulation, it would be possible to reproduce interesting abilities similar to human ones~\cite{kaufmann1979machines}~\cite{haugeland1989artificial}. The symbol-processing paradigm or cognitivist paradigm sees the cognition as pure computation. In other word, the actual intelligence process is the abstract algorithm or the program doing calculus. Eventually, researchers following this paradigm no longer saw the physical incarnation as a relevant component. Cognitive and computationalists hypotheses stating that the thought is reducible to a set of symbolic calculations are being established~\cite{fodor1987psychosemantics}. The body, for its part, is forgotten, irreparably separated from the mechanisms of intelligence~\cite{kaplan2008corps}.
In addition, the robot body became a handicap, which often ruins the efficiency of algorithms and programs created by AI researchers. Indeed, the real world body is non-perfect, there is some noise on sensors acquisition, there is gravity, friction and inertia acting on actuators, and the environment is always changing and unpredictable.

To overcome these issues of real world applications, the other side of the robotics community, still interested in the hardware challenges strives to design more reliable and powerful robots which can react as fast and as close as possible to the model used for its control. To do that, it is needed to have way more precise sensors and powerful enough actuators to overcome inertia and mechanical friction. Thanks to these work on hardware, industrial robots became more and more fast and precise, enough to outclass any human on specific assembly tasks.

However, even with really efficient robots, artificial intelligence failed to show results comparable with the expectations researchers and society had. Robots are able to solve incredibly complex task such as chess game or able to achieve highly precise tasks in manufacture but require perfectly controlled and predictable environment. Going outside this known environment seems impossible to program and none of them is able to act fluently in the real world.
But unlike virtual worlds, the real world is challenging in various ways. It is not possible to enable omniscience: we have not access to the knowledge of the whole world state and parameter, the measures a robot can take are limited, take time and are noisy while the action took are always different that. Finally, the world state is never clearly defined, based on precise discrete states; "the weather is never simply goof or bad"~\cite{piefer06}.

While classical approach known great successes to solve abstract problems such as chess game, search engine, text processing, however it failed in the understanding of natural forms of intelligence which requires a direct interaction with the real world. This is especially the case when we think in the current state of the art for interaction with human (natural language) or object (grasping) and the locomotion in an open environment (walk, run, ride a bicycle).

\subsection{Emergence of the embodiment paradigm} % (fold)

Stuck with these major issues raised by acting in the real world, a kind of crisis of the artificial intelligence happened in the 1980's and the cognitivist paradigm was questioned. While some researchers of the field introduced new tools such as neuronal networks, another part questioned the "cognition is computation" approach and the irrelevance of the body.
Thanks to researchers such as Rodney Brooks~\cite{brooks1986achieving}, Rolf Pfeifer~\cite{pfeifer2001understanding} or Luc Steels~\cite{steels1995artificial}, a novel paradigm emerges: the cognition needs a body to think. The embodied artificial intelligence rejects the symbolic approach and postulates that it is not possible to have intelligence without the body and the environment~\cite{pfeifer2001understanding}. Rather than postulating there is a hierarchical structure in which the brain control the body, the new theory focuses on the interaction between the two systems, even for mathematical thinking we could assume is purely abstract~\cite{lakoff2000mathematics}.

Following this paradigm, several researchers tried to tackle challenges in which the classical cognitivist approach failed i.e. the understanding of natural forms of intelligence, which requires a direct interaction with the real world. The locomotion is a great example of task where the classical robotic approaches did not get expected results.

Animals are incredibly skilled. Even if we consider insect with a brain thousand of times smaller than the human one, theirs abilities to move in an open world is just incomparable with the most advanced current robots. One important reasons for this is that in the classical view, the ability to figure out where you are is based in detailed inner models or representations either have to be programmed into the robots or learn by interacting with the environment and continuously updated. The more complex these models are, the more effort is needed to acquire the relevant data to maintain them leading to major problem when learning task in a highly dimensional spaces (plein de REF). Brooks even argued that intelligence always requires a body and that we should forget about complex internal representations and models of the outside world; that we should not focus on sophisticated reasoning processes but rather capitalizer on the system-environment interaction~\cite{brooks1991intelligence}~\cite{brooks1995intelligence}. Then he started to work on the insect locomotion because if we understand the insect-level-intelligence it will be much easier and faster to understand and build human-level intelligence~\cite{brooks1996prospects}.

% \textbf{TODO: petite review du boulot de brooks avec les insects}

Exploring the role of the morphology and how it shapes the ways we think appears a fascinating open field. Indeed, exploring the interaction between body properties and cognition could lead to both a better understanding of animal's behaviours (human being in particular) and to build robot more adapted and robust to an open environment with unpredictable interaction.

Thus an interesting evolution of the last decades is the demonstration of the importance of the morphology for sensorimotor control, cognition and development. The research community exploring the embodiment paradigm has grown but surprisingly not as much we could imagine with classical paradigm fails. However, new work arises introducing new principles we will describe in this chapter such as morphological computation, compliance or ecological balance, emergence.

In the context of this thesis we will talk about intelligence with the meaning, ability to move in a natural environment and interact with people and objects.



\section{Exploring the role of robot morphology} % (fold)

The achievement of robust locomotion in a natural environment is one of the major current challenge for robotics researchers. For decades and it is still mostly the case, the challenge of locomotion for robotic agent was only tackles through symbolic abstract and complex computation of internal model and representation of the world. The body is reduced to a noisy interface between the abstract algorithm and the real world.
However, regarding the nature, it appears obvious that an animal morphology deeply change the way it can act in its ecological system and so it has evolved trying to optimize its body properties.

For some reasons, in the robotics and artificial intelligence field the link between the body properties and the ability for a robot to move in an ecological environment does not seems as obvious. The fact that the ability to act and achieve complex task are due to the brain computation is so deeply grounded that it event affects the general public.

However, while we can think there are indeed calculi necessary to achieve complex tasks, there is no reason it should be explicit with a precise internal models or representations of the physical world while it could be directly done through body properties.

Therefore since the 80's, considering the robot morphology defined as \emph{any characteristic which defines the physical structure of the robot such as link sizes, number of links, joint characteristics, mass distribution, actuator characteristics, material properties, sensor characteristics and sensor placements}~\cite{paul2006morphological}, novel research topics appeared exploring the role of robot body morphology towards the achievement of complex tasks in natural world, especially robot locomotion.


\subsection{Morphological computation principle} % (fold)

Introduced by Rolf Pfeifer~\cite{pfeifer2005morphological}, the morphological computation principle states that a part of the computation needed in the achievement of a given task can be done implicitly through the interaction of physical form with the ecological niche environment.
Indeed, the morphology of a robot affects its control, because it not only determines the behaviours that can be actually performed, but also the amount of control required to achieve correctly this behaviour.

A great illustration is the achievement of the fly by plane. Most of the control permitting the plane to fly is done by the interaction between the wings and the air. Indeed in a plane, the shape of its wings is critical. Their profile generates the lift while their shape and position determine the stability of the fly (see focus box). Actually the control is only necessary to operate the fly between two positions.

In this way, the interaction relationship between sensory-motor apparatus, morphological properties, environment and control is of prime importance. This relationship was first observed and characterized by Pfeifer as the morphology and control trade-off ~\cite{pfeifer2001understanding}, but the mechanisms underlying this relationship have been unclear. The fact that simple physical interactions give rise to computation indicates the theoretical possibility for the dynamics of the morphology to play a computational role in the system, and thereby to subsume part of the role of control~\cite{paulinvestigation}.


\subsection{Passive and semi-passive walking} % (fold)

The role of morphology in robot biped locomotion has been particularly explored through the research on passive dynamic walkers~\cite{wisse2007passive}.

Bipeds walking on slightly inclined plane appeared as toys in the early twentieth century. Their legs are straight and they rock on sides to allow feet lift off the ground. The analysis of the behaviour of such systems, purely passive, is much more recent. Indeed, an advantage of such object is their low energy consumption. The energy supplied to the system comes from the variation of potential energy due to the slope. It compensates the lost energy during impact.

The unipodal transfer movement is similar to a passive pendulum and arises from the correct combination of an initial pulse and coupled gravity-inertial effects. The behaviour of the walker is then to the inverted pendulum.

In the early 90's, Tad McGeer, coming from an aeronautic background, formalizes the idea of a compass biped with free articulation by the concept of synthetic wheel~\cite{mcgeer1990passive}.
The dynamics of the system is formalized by an equation of motion linearized close to an average vertical position of the legs and an equation of shock upon contact foot/ground modelling energy dissipation~\cite{mcgeer1992principles}.

\begin{figure}[tb]
\centering
    \subfloat[][]{\label{fig:}\includegraphics[width=0.4\linewidth]{simple_leg_wheel.jpg}}
    \hfil
    \subfloat[][]{\label{fig:}\includegraphics[width=0.4\linewidth]{simple_leg_wheel_trajectories.jpg}}
    \caption{The simplest of walking models is the synthetic wheel, a biped with straight legs and semi-circular feet (a). The stance leg rolls forward steadily like a spoke in a wheel, while the free leg swings ahead like a pendulum. Support is transferred between legs when their speeds and angles match (b). The cycle is naturally stable and will repeat continuously, thus synthesizing the motion of ordinary wheel~\cite{mcgeer1992principles}.}
    \label{fig:synthetic-wheel}
\end{figure}


The tuning of initial conditions conducting to a passive movement is performed numerically, after a step, the robot should be back to its initial state. This model allows to obtain a robot cyclic walking gait completely passive (without motorization) and stable on an plan, slightly inclined by a few degrees. The potential energy gained during the descent exactly compensates energy dissipated during impacts.


\subsubsection{Passive walkers} % (fold)

\begin{figure}[tb]
\centering
    \subfloat[][Tad McGeer with his prototypes]{\label{fig:tad_mcgeer}\includegraphics[width=0.49\linewidth]{tad_mcgeer.jpg}}
    \hfil
    \subfloat[][Passive walker robot]{\label{fig:mcgeer_walker}\includegraphics[width=0.49\linewidth]{mcgeer_walker.jpg}}
    \caption{}
    \label{fig:mcgeer_work}
\end{figure}

Tad McGeer has also showed that passive walking can be obtained on a bi-articulated robot~\cite{mcgeer1992principles}. An appropriate feet shape and a judicious mass distribution allow the generated footstep combining a forward pendulum swinging movement on its stance leg and a swing with spontaneous flexion of the leg transferred. To make this motion possible, a device must prevent against leg bending during the stance phase.
The dynamic behaviour is mainly determined by three dimensionless parameters: the length ratio, the mass ratio and the slope of the planar support ~\cite{Garcia1998}.


\subsubsection{Semi-passive walkers} % (fold)
Passive robots are limited to walking on inclined ground, they cannot have a passive trunk and finally they are locally stable robots, the limit-cycle attraction domain is small.

\begin{figure}[tb]
    \begin{center}
        \includegraphics[width=0.99\linewidth]{cornell_biped_series.jpg}
    \end{center}
    \caption{Caption here}
    \label{fig:figure1}
\end{figure}

Thus this work has been pursued with the apparition of semi-passive walkers combining both specific passive properties and low power actuation to increase their robustness~\cite{Anderson2005}. We can note the work of Collins~\cite{collins2005bipedal} which explored the case of semi-passive 3D biped robot. Its morphology is based on particular mass distribution, knee locking, round feet and springs on the legs to generate an efficient walking gait while keeping its lateral and frontal balance. The concept of 3D semi-passive robot has been pushed even further with the realization of a complete humanoid robot with trunk, arms and head: the robot Denise~\cite{wisse2005three} and Flame presented in~\cite{Hobbelen2008}.

% \begin{figure}[tb]
%     \begin{center}
%         \includegraphics[width=0.6\linewidth]{comparison_cost_transport.jpg}
%     \end{center}
%     \caption{Caption here}
%     \label{fig:figure1}
% \end{figure}


\subsection{Emergence of complex behaviours} % (fold)

Finding the rules that can lead to a desired behaviour is more difficult than explaining an actual complex behaviour when we can observe an agent interacting with its environment. Because the fact that the behaviour itself cannot be preprogramed but is always the result of an agent-environment interaction, we must design for emergence rather than directly for a specific behaviour~\cite{Pfeifer06}. It is called the design of emergence~\cite{Steels1991emergence} which remains an open question how this can be done systematically. At the moment, design for emergence is rather an art than a hard-core engineering discipline.

It is precisely in the art field that we find one of the most fascinating examples. Theo Jansen is a kinematic sculptor. This artist playing with field frontiers, between engineering, research and art is the designer of the sand beasts (see \figurename~\ref{fig:theo_jansen_beast}).

\begin{figure}[tb]
\centering
    \subfloat[][]{\label{fig:theo_jansen_beast}\includegraphics[width=0.99\linewidth]{theo_jansen_beast.jpg}}\newline
    \subfloat[][]{\label{}\includegraphics[width=0.32\linewidth]{strandbeest_theory.jpg}}
    \hfil
    \subfloat[][]{\label{}\includegraphics[width=0.32\linewidth]{strandbeest_motion.jpg}}
    \hfil
    \subfloat[][]{\label{}\includegraphics[width=0.32\linewidth]{strandbeest_leg_element.jpg}}
    \caption{}
    \label{fig:beast_mechanism}
\end{figure}


These giant structures move using a really clever mechanisms composed of eleven rods which lengths have been tuned by numerical optimization. This system produces a walking motion(see \figurename~\ref{fig:beast_mechanism}) with a center of rotation always remaining at the same level, for this reason Theo Jansen likes to say he "reinvented the wheel" but adapted to the environmental niche of his creatures .i.e. the beach.

Since the beginning of this work, Theo Jansen created dozens of creatures, being more and more evolved. However, the very basic mechanism remains the same, both simple because it is composed by only one degree of freedom, and complex because the length ratio between the eleven rods are critical and must be equal to specific numbers. Thus, using only really basic material, electric plastic tubes, Theo Jansen created multi-legged creatures capable of moving in the sand, powered by the wind~\cite{jansen2007theo}.




Evolution of his work, conducted to several improvements. In this video: \url{http://youtu.be/rWbU3eV4ZpQ}, 72 legs moving at the same time using one cranks. But he also extended the mechanism to add a kind of independence. For instance, he added lemonade bottles to store energy. These bottle are used as pressure tank filled using pumps powered by the wind. Beasts can use this stored energy in case the wind fade away.

Also, a natural enemy of these beasts is the sea, using the same basic material, Theo Jansen created sensors able to detect the water and reverse the way beast move. The same principle allows also these beast to avoid obstacle.

Thus the work of Theo Jansen goes beyond the kinematic art and is really instructive for the robotic and IA research fields. Indeed, thanks to a specific morphology adapted to their environmental niche, his creatures are able to act autonomously and "survive" in the real world. No computation, no abstraction, the appeared intelligence of these creatures only came from a direct interaction between their particular morphologies and the environment. Based on low cost materials, yet clever mechanisms, his work is a meaningful proof of concept showing how the morphology of an agent can lead to the creation of complex behaviour such as ones could called them "intelligent".


% \subsection{Bio-inspiration} % (fold)

% Toward the understanding of natural form of intelligence and the achievement of robust robotics, several researcher have been interested by the insect and animals kingdom. Indeed, it appears that nature has created a wide variety of very efficient organisms.


\subsection{Compliant robotics} % (fold)
The compliance describes the stiffness of a system. It is mostly used in robotics to describe how an actuator or a mechanical part reacts to external forces when trying to reach a position. The actual output of compliant actuator will be modified by its interaction with the environment while a rigid actuator will enforce the output to be the desired one. Compliance can also be achieved by the uses of soft materials for the mechanical structure (also called soft robotics), therefore the link or the shape of the robot can be deformed by its interaction with the environment.

Several projects have already shown the importance of an adequate compliant morphology to achieve complex sensory-motor such as legged locomotion in complex environment with relatively few control.  This is illustrated by the quadruped Big Dog whose compliance relies on hydraulic actuators~\cite{raibert2008bigdog} as well as the RHex robot~\cite{saranli2001rhex} using six compliant legs. Both demonstrate impressive adaptability and crossing behaviour over rough-terrain.

\begin{figure}[tb]
\centering
    \subfloat[][]{\label{}\includegraphics[height=5cm]{rhex.jpg}}
    \hfil
    \subfloat[][]{\label{}\includegraphics[height=5cm]{raptor-robot.jpg}}
    \caption{}
    \label{fig:compliant_robot}
\end{figure}

Also numbers of humanoid robots have shown the importance of compliant structure for human robot interaction. As an example Acroban~\cite{Ly2011bio} \cite{Oudeyer2011}, which the compliant structure of its vertebral column and legs was shown to permit a self-organized physical human-robot interface allowing non-expert users to lead the robot by the hand.

Furthermore it has been shown that the compliance of the body explains the dynamics of walking and running~\cite{Geyer2006} \cite{iida2007bipedal}, and appears as key-feature toward the achievement of the current faster robots both for quadripedal (DARPA Cheetah robot\footnote{\url{http://youtu.be/d2D71CveQwo}}) and bipedal running (see raptor robot\footnote{\url{http://youtu.be/lPEg83vF_Tw}}).



\section{Conclusion} % (fold)

Researchers have explored several aspects showing the interest of adapted morphology such as morphological computation, passive dynamics or compliance toward this goal. In addition, we could complete this review about the role of morphology with the ecological balance principle~\cite{pfeifer2005new}, bio-inspiration~\cite{scarfogliero2009use} \cite{pfeifer2007self} or the coupling of adequate morphologies with central-pattern generators that has been shown to generate robust locomotor behaviour~\cite{ijspeert2007swimming} \cite{steingrube2010self}.

The work presented in this chapter shows the recent awareness in the robotics field of the importance of the morphology. It appears more and more clear that the achievement of robust robotic requires the understanding of the interaction between the robot morphology, control and environment. Yet as the

However, as shown in the diversity of projects, the role of morphology is still an open research field to explore. For this purpose, an abstract robot is not sufficient and it is necessary to have a real robotic platform for experimenting. This will be the subject of the next chapter, which will discuss the technological solutions for robotics experimentation.
