% !TEX root = ../../thesis.tex

% \cleardoublepage
% \newpage
% \thispagestyle{plain}
% \mbox{}
\includepdf{/Users/matthieulapeyre/Documents/phd_thesis/media/thebeast.pdf}

\chapter{Robot morphology: some facinating work} % (fold)

\section{Morphological computation} % (fold)

Works of Tad Mcgeer, theo jansen

\subsection{Locomotion} % (fold)

\subsection{Robustness} % (fold)


\section{Morphology and cognition} % (fold)

\subsection{Data filter} % (fold)

exemple de l'oreille

\subsection{Control stuff} % (fold)

La compliance c'est chouette


Scientific study of the role of morphology in sensorimotor control and cognition: in Robotics (McGeer, Pfeifer and co.), in relation with Cognitive Science (e.g. http://www.pyoudeyer.com/IEEETAMDOudeyer10.pdf ) and animals (e.g. work of Robert Full)

\textbf{Object}: In this chapter, we will present a review of different research which has already explored the role of the mophology


\textbf{Conclusion}: The body can definitely take in charge a part of the complexity. And we need to continue studying it by EXPERIMENTATIONS. Even if simulator can be complementary.
% chapter exploration_of_the_morphology_role (end)

