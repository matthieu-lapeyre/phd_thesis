% !TEX root = ../../thesis.tex

\chapter{Introduction} % (fold)


\section{Robotics, an expected societal (r)evolution} % (fold)

Robotics is predicted to have a major impact on our societies. 
A lot of research is done about for elder assistance. Especially in Japan where the aging population will pose problems in the years to come.

Also while robots have already had a strong impact on the industrial process. New applications are arising to allow close collaboration with workers.

Robots are also expected to intervene during disaster and help to help rescuers in their actions.

We can cite also the companion robots... games, assistance, complementary to smartphones and domotics. 


\emph{Je ne sais pas où parler du role de la robotique pour faire autre chose que de la robotique: sciences humaines, comprendre l'homme, education, art, ...}

\section{Major scientific challenges still have to be addressed} % (fold)

Even if economic agitation arises since Google made huge investments to acquire seven of the most important robotics start-up. There are still major challenge to be solved before robots may have an actual impact on our daily life.


\subsection{General} % (fold)
Complicated robotics stuffs we do not care in this thesis such as perception, vision, communication, manipulation, ...

\subsection{A multidisciplinary science} % (fold)
An autonomous robot have a mechanic body, an electronic architecture and use computers for the control. 

Robotics is intrinsically an association of different sciences coming from a large spectrum of domains. It is both what makes robotics so fascinating and so challenging. While we can contribute to the robotics from several horizon and have the possibility to create multidisciplinary collaboration, we have also to face technical issues coming from the application of all these sciences.

As researcher, we  are often confronted to technical problems not directly link to the problematic we are interested in. This is especially the case when we set up experiments. A lot of time is consumed doing technical debug. All these problems tend to increase time for the achievement of research milestones.


\subsection{The real world} % (fold)

The current state of the arts is effective to control with high precision and velocity complex multi-link robotics arms but is not robust under unpredictable events. Thus going outside the lab or a factory is still really difficult for robots. 

\subsubsection{Walking around in an unknown environment} % (fold)
While first robotics cars are traveling along some American roads (flying drone also), the legged locomotion, especially the biped one, is not yet enough robust to be efficient on unspecified terrain. An example was the first round of the DARPA Challenge which showed how far we were from biped robots actually efficient in the real world.


\subsubsection{Interaction with real humans} % (fold)
In addition to moving in an unknown environment, robots will have to interact with real people. 
Physical and Social ... 


\subsection{The need of long term experiments} % (fold)
After years of research, only few experiments and show how far we were from a robot, enough autnomous for beign useful



\section{The robot morphology: the interface with the real world} % (fold)

\subsection{Effective locomotion needs adapted morphology} % (fold)



% #TASK
