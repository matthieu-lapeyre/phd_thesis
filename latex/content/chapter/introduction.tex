% !TEX root = ../../thesis.tex

\cleartoleftpage
\includepdf{../media/chapter_illustration/poppy_prairie_2.pdf}

\chapter{Introduction} % (fold)


In the Inria Flowers team\footnote{\url{flowers.inria.fr}}, we are interested by the study of mechanisms that can allow robots and humans to acquire autonomously and cumulatively repertoires of novel skills over extended periods of time.

This includes mechanisms for learning by self-exploration, as well as learning through interaction with peers, for the acquisition of both sensorimotor skills (e.g. locomotion, affordance learning, active manipulation) and social skills (e.g. grounded language use and understanding, adaptive interaction protocols, and human-robot collaboration).

An interesting evolution of the last decades is the demonstration of the importance of the robot morphology for sensorimotor control, cognition and development (\cite{kaplan2008corps} \cite{steels1995artificial} \cite{Pfeifer06}). Indeed, a robot behaviour cannot be preprogramed. The actual behaviour is always emerging from a complex interaction between the control algorithm, the robot morphology and the environment~\cite{Steels1991emergence}. Moreover, it is clear that an adapted robot morphology using specific properties can greatly reduce the complexity of a given task by ensuring implicitly a part -or the integrity- of the control required~\cite{pfeifer2005morphological}.
Finally, as Rodney Brooks argued, \emph{the world is its own best model}~\cite{brooks1991intelligence} and simulators cannot realistically handle the complexity of the real physic with multi-point contacts, soft materials compliance, frictions or unpredicted multi-modal interactions.


Exploring mechanisms of acquisition requires to take into account the robot morphology to be complete.
Therefore, we \textbf{should consider the robot morphology\footnote{The robot morphology is defines as any characteristic which defines the physical structure of the robot such as link sizes, number of links, joint characteristics, mass distribution, actuator characteristics, material properties, sensor characteristics and sensor placements~\cite{paul2006morphological}} as an experimental variables~\cite{kaplan2008corps} that can be tuned and conduct experiments in the real world}.

Where it is straightforward to explore and experiment variation of several parameters in software (e.g. algorithms, simulator), experimenting variation of the morphology on a real robot is much more challenging:

\begin{enumerate}
    \item how can we have an experimental robotic platform allowing both to change easily and quickly its morphology while acting robustly in the real world ?
    \item how can we make this work, mainly hardware, diffusible and reusable in the research community ?
\end{enumerate}

Unfortunately current robotic platforms are not suitable to address such challenges.

On one hand, commercial robots such as Nao \cite{gouaillier2008nao}, Darwin Op \cite{ha2011development}, Nimbro Op \cite{schwarznimbro} or iCub \cite{metta2008icub} are easily accessible and easy to use. Yet they provide a "traditional" morphology (e.g. limited compliance, rigid torso, big feet, over actuated) and they do not permit the modification of their morphology. Moreover in most case, they are not open source and/or the hardware is to complicated/expensive to be modified.

On the other hand, lab prototypes are mainly handcrafted and specifically tuned which make them almost impossible to be reproduced in another lab.

The main issue of these robots are the chosen approaches and technologies used to design and produce them. Indeed, the classic way to design and produce robot is a complicated, time-consuming and an expensive process involving specific upfront tooling and complex manufacturing processes.


Within this context we decided to create a whole new humanoid robot called Poppy. This humanoid robot was designed to easily and quickly conduct scientific experiments on sensorimotor learning, exploring morphological properties, and human-robot interaction. As an experimental robotic platform, Poppy was designed to be \textbf{affordable}, \textbf{lightweight}, \textbf{robust and safe}, \textbf{easy to use}, \textbf{highly-hackable} and \textbf{fast and easy to duplicate or modify} with the goal to be easily reproducible and used by other lab thanks to an open source distribution (hardware and software). This was achieved thanks to 3D printing techniques, affordable off-the-shell components and optimized modular design.


\section*{Proceeding of this thesis} % (fold)


This thesis will be presented along 4 parts.

In the first part, we will present the related work about the role of morphology.

The related work will be composed of 3 chapters covering important aspect associated with the Poppy project. First, we will discuss important research work on the robot morphology that show how essential the robot morphology is toward

...








