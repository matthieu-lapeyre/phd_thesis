% !TEX root = ../../thesis.tex

\chapter{Introduction} % (fold)


\section{Robotics, an expected societal (r)evolution} % (fold)

Robotics is predicted to have a major impact on our societies.
A lot of research is done about for elder assistance. Especially in Japan where the aging population will pose problems in the years to come.

Also while robots have already had a strong impact on the industrial process. New applications are arising to allow close collaboration with workers.

Robots are also expected to intervene during disaster and help to help rescuers in their actions.

We can cite also the companion robots... games, assistance, complementary to smartphones and domotics.


\emph{Je ne sais pas où parler du role de la robotique pour faire autre chose que de la robotique: sciences humaines, comprendre l'homme, education, art, ...}

\section{Major scientific challenges still have to be addressed} % (fold)

Even if economic agitation arises since Google made huge investments to acquire seven of the most important robotics start-up. There are still major challenge to be solved before robots may have an actual impact on our daily life.


\subsection{General} % (fold)
Complicated robotics stuffs we do not care in this thesis such as perception, vision, communication, manipulation, ...

\subsection{A multidisciplinary science} % (fold)
An autonomous robot have a mechanic body, an electronic architecture and use computers for the control.

Robotics is intrinsically an association of different sciences coming from a large spectrum of domains. It is both what makes robotics so fascinating and so challenging. While we can contribute to the robotics from several horizon and have the possibility to create multidisciplinary collaboration, we have also to face technical issues coming from the application of all these sciences.

As researcher, we  are often confronted to technical problems not directly link to the problematic we are interested in. This is especially the case when we set up experiments. A lot of time is consumed doing technical debug. All these problems tend to increase time for the achievement of research milestones.


\subsection{The real world} % (fold)

The current state of the arts is effective to control with high precision and velocity complex multi-link robotics arms but is not robust under unpredictable events. Thus going outside the lab or a factory is still really difficult for robots.

\subsubsection{Walking around in an unknown environment} % (fold)
While first robotics cars are traveling along some American roads (flying drone also), the legged locomotion, especially the biped one, is not yet enough robust to be efficient on unspecified terrain. An example was the first round of the DARPA Challenge which showed how far we were from biped robots actually efficient in the real world.


\subsubsection{Interaction with real humans} % (fold)
In addition to moving in an unknown environment, robots will have to interact with real people.
Physical and Social ...


\subsection{The need of long term experiments} % (fold)
After years of research, only few experiments and show how far we were from a robot, enough autnomous for beign useful



\section{The robot morphology: the interface with the real world} % (fold)

\subsection{Effective locomotion needs adapted morphology} % (fold)

\section{IROS} % (fold)
Research in humanoid robotics has been thriving in the recent years \cite{hirai1998development} \cite{kaneko2008humanoid}, both due to the predicted relevance of humanoid robots for personal and assistive robotics \cite{tapus2007socially}, and due to fundamental scientific questions they raise with regards to biped locomotion and full-body physical interaction with the environment. Indeed, biped robots need to be able to move robustly and efficiently in human environments, which include the ability to keep stability when unpredictable physical contact with humans happens. At the same time, these robots need to be capable of rich and safe social and physical interaction with humans, and to adapt to the behavior and preferences of each particular user.

We should not only try to solve these challenges through artificial cognitive intelligence but also through body intelligence \cite{Pfeifer06}. On one hand, a way to permit robots to adapt their behaviors to unknown environments is to provide them with control algorithms which can be updated with learning algorithms based on social guidance \cite{billard2008robot}, or on autonomous self-exploration \cite{Baranes2012RAS}\cite{lapeyre2011maturational}. On the other hand, a part of the computation needed for such adaptation could also be done through the intrinsic mechanics and electronics of the robot, thus providing effective and hyper-responsive reactions while simplifying the algorithms of the different behaviors. This role of morphology has been called morphological computation \cite{pfeifer2005morphological}, as the body of the robot becomes a form of information processing structure, capable of supporting a part of the computation necessary to achieve sensorimotor tasks to simplify or make it more robust to external disturbances \cite{pfeifer2005morphological}\cite{Pfeifer07}. The actions or reactions of the physical body also have the advantage of being direct without latency due to a controller, as opposed to CPU computed reactions which often require high-cost hardware in order to respond fast enough and reduce modeling errors.


The work presented in this article takes place within a research program exploring which mechanisms can allow humanoid robots to acquire sensorimotor and social skills in a life-long manner, and through self-exploration and social interaction with non-technical users \cite{Weng01}\cite{lungarella:03a}\cite{Oudeyer07}\cite{oudeyerEncyclo11}\cite{Oudeyer13}. A central vision of this research program, deeply inspired by infant learning and development, is that life-long skill learning in the real world can only effectively happen if statistical inference is guided by strong constraints, in particular related to the physics of the body (their material, their geometry and the evolution of this geometry as the body grows \cite{Berthouze04}\cite{Baranes11a}\cite{lapeyre2011maturational}) and to the social environment \cite{billard2008robot}\cite{argall2009survey}. Indeed, typical humanoid bodies are high-dimensional, which is extremely challenging for acquiring sensorimotor controller. Body intelligence and social guidance have been argued to facilitate and considerably guide the learning and development of sensorimotor skills in these complex spaces \cite{Oudeyer13}.
Within this context, we present in this article the humanoid robot Poppy, which elaboration was done to address the following design goals:
% #TASK
