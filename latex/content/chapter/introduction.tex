% !TEX root = ../../thesis.tex

\chapter{Introduction} % (fold)


\section{Robotics, an expected societal (r)evolution} % (fold)

For decades, Robotics was predicted to play a major role in our society, yet it turned as a rampant subject often overtaken by unexpected innovations such as the incredible expansion of the Internet in the last twenty years.
However the current international activity on the robotic market tends to show a potential societal impact in a near future.

\subsection{A growing market} % (fold)

\emph{Stuff about the current activity on the market showing an expansion of robotic applications for large public applications in the 5 incoming years.}

\url{http://www.logisticsmgmt.com/article/robot_report_predicts_significant_growth_in_coming_decade}
\url{http://www.computerworld.com/s/article/9054701/Personal_robot_market_expected_to_balloon_to_15B_by_2015}

Google bought the most innovative robotics companies such as Boston Dynamics, Shaft, Meka or Deep Mind.
Drone and stuff ...
google car, ardrone, shipping with drone by amazon etc ...

The domestic robot, in particular vacuum cleaner are under expansion with millions of units sold from various brand.

\subsection{industrial} % (fold)
Also while robots have already had a strong impact on the industrial process.
New applications are arising to allow close collaboration with workers.
Industrial robots are becoming accessible to small company with robot such as Baxter.

The robotic can permit developed country to re-locate industrial factories while the human cost will be reduced by automatic manufacturing process using robot massively.


\subsection{Assistance} % (fold)
Currently, a lot of research is done to tackle issues raised by the aging of the population.
Especially in Japan, elder assistance is becoming a priority.


\subsection{Education} % (fold)


\subsection{other stuffs} % (fold)
\emph{Je ne sais pas où parler du role de la robotique pour faire autre chose que de la robotique: sciences humaines, comprendre l'homme, education, art, ...}

Plus proche de nous, dans les annees 1940, Grey Walter, chercheur en neurophysiologie, reconnu pour son travail sur l’electroencephalogramme et pour les nombreux progres que lui et son equipe y apporterent, etait convaincu que de developper de nouveaux systemes de visualisation et d’analyse etait sans doute necessaire mais non suffisant pour comprendre le cerveau en action. Il construisait des petits robots mobiles pour montrer que, au-dela de l’organisation cerebrale, ce sont aussi les caracteristiques anato- miques et physiques du corps qui determinent le comportement (Walter, 1967; Cordeschi, 2002). Aujourd’hui les experiences « robotiques » s’invitent dans de nombreux domaines, depuis les neurosciences (Edelman, 2007) jusqu’a la psychologie du develop- pement (Revel, Nadel, 2007) ou meme la linguistique (Kaplan, 2001 ; Steels et al., 2002).

\section{Interaction with the next industrial revolution} % (fold)
Following the Internet emergence, an industrial revolution is in progress~\cite{anderson2012makers}.


\section{Major scientific challenges still have to be addressed} % (fold)

Even if economic agitation arises since Google made huge investments, there are still major challenges to be solved before robots may have an actual impact on our daily life.


\subsection{General} % (fold)
\emph{Complicated robotics stuffs we do not care in this thesis such as perception, vision, communication, manipulation, ...}

\subsection{A multidisciplinary science} % (fold)
An autonomous robot have a mechanic body, an electronic architecture and use computers for the control.

Robotics is intrinsically an association of different sciences coming from a large spectrum of domains.
It is both what makes robotics so fascinating and so challenging.
While we can contribute to the robotics from several horizon and have the possibility to create multidisciplinary collaboration, we have also to face technical issues coming from the actual application of all these sciences.

As researcher, we  are often confronted to technical problems not directly link to the problematic we are interested in.
This is especially the case when we set up real world experiments.
A lot of time is consumed doing technical debug.
All these problems tend to increase time to achieve research milestones.


\subsection{The real world} % (fold)

The current state of the arts is effective to control with high precision and velocity complex multi-link robotics arms but is not robust under unpredictable events.
Thus going outside the lab or a factory is still really difficult for robots.

\subsubsection{Walking around in an unknown environment} % (fold)
While first robotics cars are traveling along some American roads (flying drone also), the legged locomotion, especially the biped one, is not yet enough robust to be efficient on unspecified terrain.
An example was the first round of the DARPA Challenge which showed how far we were from biped robots actually efficient in the real world.


\subsubsection{Interaction with real humans} % (fold)
In addition to moving in an unknown environment, robots will have to interact with real people.
Physical and Social ...


\subsection{The need of long term experiments} % (fold)


\section{The robot morphology: the interface with the real world} % (fold)

A major scientific results in the field in the last decade is the identification of the importance of the body to simplify motor control problems for bipedal locomotion for example, and to create intuitive interactions between people and robot.

