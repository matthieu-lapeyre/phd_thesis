% !TEX root = ../../thesis.tex


\chapter{Discussion} % (fold)

Here we will applied it to design a whole new humanoid but this methodology is adaptable to any kind of robot.

\section{Mouvement makers} % (fold)

http://www.withoutmodel.com/lomig-unger/pourquoi-renault-sinteresse-aux-fab-labs/

\section{Cumulative Science} % (fold)

\subsection{Modern tools for contemporary science} % (fold)

Discussion around the new tools allowing the scientific impact.
Discussion about the role of social networks for mediation

\subsection{Going outside the laboratory} % (fold)
Going outside the Lab with Fablab and living lab.
More multi-disciplinary collaboration, work with artist, designer, illustrator.

% subsection going_outside_the_laboratory (end)

\subsection{Sharing more than papers} % (fold)

The paper system came from a old way to publish science. Now there is internet, there is no raison to pay hundreds of dollars for uploading a paper.

Open source


\section{Expected impacts} % (fold)

\subsection{Development tool and benchmarking platform for Science} % (fold)

\subsection{Poppy for education} % (fold)

\subsection{Poppy and artists} % (fold)


\subsection{Side effects} % (fold)
People building/using Poppy will face the current robotics challenge. Through the construction of Poppy, a robotic mediation can occur leading to the understanding of the limitation of the curent state of the art.

It can lead to the transmission of an alternative vision of the robotics field, more as a way to express creativity and explore the complexity of nature, than a terminator work in progress.


% chapter the_importance_of_scientific_dissemination (end)
