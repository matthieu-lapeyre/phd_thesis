% !TEX root = ../../thesis.tex
\newpage
% \thispagestyle{empty}
\mbox{}
% Photo Yves Gellie
\cleartoleftpage
\includepdf{../media/chapter_illustration/robot_setup.pdf}
\chapter{Review of experimental methods} % (fold)
\label{cha:experimental-methods}

\cleanchapterquote{The world is its own best model.}{Rodney Brooks}


\section{Introduction} % (fold)
\label{sec:platform-introduction}

As we showed in the previous chapter, an interesting evolution over the last decades in the robotics field has been the demonstration of the importance of robot morphology and its impact on cognition and control. This opens new horizons toward the achievement of more adapted and robust behaviour in open-ended environments with unpredictable interaction.

However as Rodney Brooks explained, exploring interaction between morphology, cognition and environment requires real world experimentation. Indeed embodied artificial intelligence~\parencite{Pfeifer07} needs to act in the real world in order for  complex behaviour to emerge. The real world includes a large number of constraints such as inertia, multi-point physical contacts, friction, multi-body dynamic and impacts which are complicated to realistically model without involving considerable engineering resources. Moreover, if one is interested in exploring robotic behaviour in an open-ended environment with human interaction, managing the unpredictability will be limited to a small subset of cases.

From another angle, "The world is its own best model" and if we can use simulation for exploring well-defined concepts, the exploration of emergent complex behaviours based on interaction between robot self-dynamics and the environment appears easier and less expensive if conducted directly in the real world.
Moreover, as Luc Steels explained~\parencite{Steels1991emergence}, actual behaviour emerges from the interaction between the controller, morphology and the environment. Some could argue that adding actual morphology and an ecological niche add unnecessary complexities to a problem we already have difficulty solving, even just in terms of the control. Yet it appears some behaviour cannot be achieved without real physical complexity. Brilliant examples are the passive/semi-passive walking robots (presented in section~\ref{sub:passive_principle}). These robots are technically rather simple, just composed of a mechanical structure with the proper link size and foot shape. The establishment of a model should be rather easy but their real dynamic is very difficult to simulate on a classical physical simulator. Indeed, all complex physical effects (e.g. shocks, friction, inertia) contribute to the achievement of passive walking.

One of the best state of the art works in this domain was achieved by Delft University with the different passive and semi passive walkers they built~\parencite{wisse2005three}. Desiring to explore biped locomotion, we had a discussion with Martijn Wisse on simulation strategies for adding semi-passive abilities to Poppy. Here is his testimony about their attempt to make their robot walk in simulation after managing to create the real one:

\begin{formal}

Even after obtaining a successful walking motion, we did not manage to create a simulation that walked successfully using the same controller parameters. We tried very hard with some of the best people, but we didn’t succeed. The reason was, I think, that our type of control (using the emergent behaviour of a set of simple reflex-like controllers) was highly sensitive to hardware effects like friction. Normally, one uses a local joint controller to make the joint follow a desired trajectory independent of the exact amount of friction. The local controller "abstracts these hardware effects away", if you know what I mean. This makes the behaviour of the whole system quite predictable. However, in our robots, we did not have this kind of abstraction as we were not following trajectories, and thus a little bit of extra friction has an effect on the entire motion.

We did spend a long time making a high-fidelity model in Adams, and also using other methods, but eventually we gave up without success.

\signed{Martijn Wisse - Associate professor at Delft University of Technology}
\end{formal}

Thus results obtained in simulators are difficult to transpose on a real platform and vice-versa. One of the main reasons seems to be the complexity of realistically handling non-perfect components e.g. non-linear friction in joints, feet/ground reactions and so on. Yet the interesting contributions of robot morphology on its behaviour are precisely those we currently have difficulties modelling correctly.

This raises a major limitation for the reproducibility of results in the scientific community. While it is rather simple to transfer experiments conducted in a simulator by sharing the software material, sharing real world experiments is far more challenging.

In this chapter, we will review the current state of the art of robotic platforms, in particular how they are made and how the results can be demonstrated or transferred in the scientific community. Yet there are many robotics platforms, from robot arms (Jako, LWR, Kuka) to wheeled platforms (Pioneer 3-AT, P3-DX) or even submarines (AQUA2). In this thesis we are particularly interested in the exploration of morphology for locomotion and interaction so we will restrain the platform review to the ones actually exploring particular morphologies and humanoid platforms.


\section{Platform exploring the role of morphology} % (fold)

As we discussed in the previous chapter, several robots have been made to explore the role of morphology, each exploring particular aspects of this challenge.

\subsection{Bio-inspired robot} % (fold)
\label{sub:compliant_robot}

The ECCE ROBOT~\parencite{marques2010ecce1} investigates the role of morphology for cognition and human-robot interaction through a bio-inspired and compliant anthropometric design, which copies the inner structures and mechanisms (bones, joints, muscles, and tendons). Thanks to a design based on polymorph mechanical structures and wire-driven actuation, they managed to produce a really complex structure mimicking both the mobility of the human upper body and the intrinsic compliant properties of the human muscular system.
While polymorph\footnote{Polymorph is a thermoplastic polymer which melts at 62\textsuperscript{o}C and consists of small off-white plastic granules. By heating these granules in hot water the user can easily melt the pellets to form a transparent flexible material. Once melted the opaque white pellets fuse together, become transparent and soften, allowing the user to form the plastic by hand into unique shapes.} is a convenient material to easily create custom shapes by hand, the diffusion of robots based on this technology is limited because this manual process cannot be reproduced outside the lab by someone else.

The Kojiro robot~\parencite{mizuuchi2007advanced} also involves a bio-inspired morphology, but unlike the ECCE1, it is a complete humanoid robot with an advanced leg design. The project aims to show the musculoskeletal humanoid’s advantages by involving many degrees of freedom and sensors, a multi-articulated spine and compliance. Moreover,
it implements the concept of modular reinforceable muscles~\cite{mizuuchi2004design}, which means the actuation required can be explored by changing the muscle unit. Each muscle unit consists of a DC motor with a gearhead, a pulley, a tension sensor using strain gauges, a thermometer, a sensor-amplifier circuit board, and a rounded outer shell.
However, while the robot seems promising for exploring both locomotion and human-robot interaction thanks to its advanced musculoskeletal system and compliant actuation, the data permitting the duplication of Kojiro has not been distributed and the structure of the robot appears really complex, with numerous components.

% Also certain humanoid robots have shown the importance of a compliant structure for human robot interaction. For example the compliant structure of the vertebral column and legs of Acroban~(\cite{ly2011bio}, \cite{Oudeyer2011}) was shown to permit a self-organized physical human-robot interface allowing non-expert users to lead the robot by the hand.


\subsection{Passive dynamic walkers} % (fold)

There are numerous passive and semi-passive dynamic walkers. We can indeed cite the one from Tad McGeer~\parencite{mcgeer1990passive}, the work of Steve Collins~\parencite{collins2001three} and Russ Tedrake~\parencite{tedrake2005learning}, the robots made in Delft University Denise~\parencite{wisse2005three} and Flame~\parencite{Hobbelen2008}. Also more recently a passive walking robot designed and built by the Nagoya Institute of Technology (Japan) walked non-stop for 13 hours and 45 minutes on a treadmill, completing some 100,650 steps over a distance of around 15.2 km. All these robots demonstrate impressive results and show the interest of using clever morphologies for the achievement of tasks as complex as bipedal walking.

However these robots are really difficult to transfer and reproduce in the robotics community.

Firstly, their mechanical structure have mechanical parts that are either handcrafted or produced with classical machining techniques based on milling or casting metal alloys. These techniques require specific upfront tooling, which makes the production of a small batch really expensive.

Secondly, while the control of this robot is rather simple, the mechanical design is far more subtle and requires expertise few people in the robotics community have. Unfortunately, the descriptions we can find in the associated papers are limited to theoretical models. It is necessary but not sufficient. Again the talk we had with Matijn Wisse is really representative of the way passive and semi-passive robot are created:

\begin{formal}
We never actually produced a high-fidelity simulation. We made very simple simulations only. From them, we learned how to tune parameters. Then, we designed the real robots, without running full-blown optimizations. Rather, we used our intuition for a large number of decisions on design trade-offs, using lessons from the simple simulations combined with other limitations such as available motors etcetera. Then, we (again) used our intuition and large amount of experience to tune the robot’s controllers, and make design improvements, until it walked.

\signed{Martijn Wisse - Associate professor at Delft University of Technology}
\end{formal}

Thus there is a big step between the model and the achievement of a functional semi-passive robot. Indeed the engineering design of such a platform has a strong impact on the achievement of passive dynamic walkers.

It is a problem for the diffusion of such an idea as the laboratory desiring to explore passive principles has to take the risk of spending time developing a passive robot without any guarantee it will succeed in finding the appropriate tuning.


\subsection{Modular robotics} % (fold)

Despite the numerous robotic platform developed, there are only a few whose hardware design can be completely reconfigured.

We can find some modular robots~\parencite{murata2007self} examples such as Molecubes~\parencite{zykov2007molecubes}, M-Tran~\parencite{murata2002m}, Superbot~\parencite{salemi2006superbot}, ATRON~\parencite{jorgensen2004modular} or Roombots~\parencite{sproewitz2009roombots}. They are independent robot modules, which can be assembled together to create various robot form and applications (see \figurename~\ref{fig:modular-robots}). However, these kinds of robot are not suitable for exploring morphological properties or creating humanoid robots.

\begin{figure}[tb]
\centering
    \subfloat[][ATRON]{\includegraphics[width=0.45\linewidth]{ATRON.jpg}}
    \hfil
    \subfloat[][Roombots]{\includegraphics[width=0.45\linewidth]{roombots.jpg}}
    \caption{Modular robots}
    \label{fig:modular-robots}
\end{figure}

Actually, to our best knowledge there is only one example of a modular kit that makes real exploration of the role of morphology possible. The Locomorph project~\parencite{locomorph} offers a multi-purpose hardware kit called LocoKit~\parencite{larsen2012locokit}). This kit uses carbon-fiber rods assembled with Locokit parts (see \figurename~\ref{fig:locokit-parts}). It allows to quickly create robots and study the impact of several morphological properties such as link length, joint stiffness or mass distribution (see \figurename~\ref{fig:locokit}). Also it permits to add spring over rods to create a linear damping system and therefore add compliance to robots.
The kit is distributed for \texteuro2500 and includes the components needed to create a quadruped robot (see \figurename~\ref{fig:locokit-example}).

\begin{figure}[tb]
\centering
    \subfloat[][]{\label{fig:locokit-parts}\includegraphics[height=5cm]{locokit-parts.jpg}}
    \hfil
    \subfloat[][]{\label{fig:locokit-example}\includegraphics[height=5cm]{locokit-example.png}}
    \caption{}
    \label{fig:locokit}
\end{figure}

It appears to be the only existing solution allowing both the exploration of the role of morphology and the transfer of results between laboratories. While being very interesting, it is also limited asthe robot created must be rod-based so multi-body articulations such as those we can find in human leg or torsos seem complicated to produce.


\section{Commercial humanoid platforms} % (fold)

Robotic prototype platforms appear to share the same issue as regards the reproducibility of science. Most of them are constructed using classical manufacturing techniques and involve a complex morphology, which makes them expensive to reproduce both in term of human resource and material cost. Finally, they are in most cases not open source and no material is shared that allows others to reproduce them easily.

The use of commercial platforms could be an alternative as they are easily available, and have a constant and reproducible morphology.

\subsection{Advanced research platforms} % (fold)

The two most famous humanoid robots are iCub~\parencite{metta2008icub} and HRP-2. These robots involve very advanced technologies:

ICub is an open source\footnote{The hardware design, software and documentation are released under the GPL license.} robot measuring about 100 cm in height for an overall weight of 22 Kg. It has 53 degrees of freedom designed specifically for manipulation and locomotion, and powered by high-ended actuators based on a harmonic drive reduction system and brushless frameless motors~\parencite{natale2013icub}. In addition, in the majority of cases torque is transmitted from the motors to the joints using steel tendons routed in complex ways via idle pulleys.

HRP-2 is a complete humanoid 150 cm in height weighing 58 kg and comprising 30 DoFs. It also involves high-end actuators with harmonic drives and cooling systems installed in both the computer and actuator drive systems to improve temperature control, yet contrary to the ICub robot, they made the choice to have the robot as stiff as possible.

Both robots involve a very complex and advanced design, which has required the work of dozens of engineers, also their production techniques make them very expensive (i.e. more than \texteuro200,000). Therefore, these robots do not permit to freely change their morphologies. Moreover, their high cost and their fragility limit the risk researchers can take in exploring behaviour in the real world.



\subsection{affordable platform} % (fold)

On the other hand, there are small and affordable commercial humanoids platforms, that are easily accessible and easy to use such as Nao \cite{gouaillier2008nao}, Darwin Op \cite{ha2011development}, Nimbro Op \cite{schwarznimbro} or iCub \cite{metta2008icub}.

\begin{figure}[tb]
\centering
    \subfloat[][Darmin Op]{\label{fig:darwin-op}\includegraphics[height=6.5cm]{darwin_op_face.jpg}}
    \hfil
    \subfloat[][Nao]{\label{fig:Nao}\includegraphics[height=6.5cm]{nao_face.png}}
    \hfil
    \subfloat[][Nimbro OP]{\label{fig:nimbro-op}\includegraphics[height=6.5cm]{nimbro-op_1.jpg}}
    \caption{Three humanoid robots made by KumoTek (USA).}
    \label{fig:kumotek_robots}
\end{figure}


\begin{description}
    \item[DARwIn Op] is an open source humanoid research platform created by the Romela lab at Virginia tech~\parencite{ha2011development} and distributed by Robotis for about \$12,000 (see \figurename~\ref{fig:darwin-op}). It is 45cm high and weighs 2.9kg, and has 20 Dynamixel MX-28 actuators (6 for each leg, 3 for each arm, and 2 for the neck). Its mechanical structure is composed of aluminium parts.
    \item[Nao] is 55cm high, 5.2kg and 25DoFs humanoid robot with a plastic mechanical structure (ABS, PA, XCF) (see \figurename~\ref{fig:Nao}). It is a very famous robot, a few thousands units of which have been sold to labs and universities. Its cost was around \texteuro12,000, but recently it was halved.
    \item[Nimbro-OP] is a tall humanoid measuring 95cm in height and weighing 6.6 kg, it has 20 powerful MX-106 and MX-64 actuators (6 per leg, 3 per arm and 2 in the neck). It costs \$20,000 and its structure is based on an aluminium and carbon composite (see \figurename~\ref{fig:nimbro-op}).
\end{description}

Yet, they provide a "traditional" morphology (e.g. limited compliance, rigid torso, large feet, over actuation) not really appropriate for exploring the interesting morphological properties we showed in the chapter~\ref{cha:morphology-review}. Also as they use classic manufacturing techniques such as metal milling and plastic casting, the modification of their morphologies is made difficult.

\section{Conclusion}

In the previous chapter, we presented several work showing the importance of robot morphology and the need to continue research in this domain. As we explained in section~\ref{sec:platform-introduction},  this research area requires areal robotic platform to explore.

There are many robotic platforms, we could have a more exhaustive description, yet the main objective of this review was to show an overview of the landscape of possibility with the current robotic platforms. In particular that we have on one hand, some prototype robotic platforms designed for specifically exploring one aspect of robotic morphology but whose design methodology strongly limits their reproduction in the robotics community, mainly because they are not open source and produced with expensive techniques.
And on the other hand, commercial robots that are easily accessible so the results should be reproducible from one lab to another. Yet the design method used also involves classical production techniques, therefore modifying their morphology would be too expensive and time consuming.

Finally, the current research practices in the Robotics field limit diffusion and the impact of contributions.
Indeed, in most cases, there is no material associated with a published paper.
This means only the theory is shared with the community but not the actual framework allowing for the results to be reproduced.

In the next chapter, we will present novel production techniques and modes of diffusion, which can solve both problems, exploring the role of morphology and reproducibility between research labs.






