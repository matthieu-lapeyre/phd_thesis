% !TEX root = ../thesis.tex
%
\pdfbookmark[0]{Abstract}{Abstract}
\chapter*{Abstract}
\label{cha:abstract}
\vspace*{-10mm}

This thesis \textbf{suggests novel approaches and design processes to create and produce robotic platforms,  the control and morphology of which can be freely explored through experimentation in the real world,  that are easy to diffuse and reproduce in the research community.} Especially, this alternative design methodology is driven by the desire to:
\begin{itemize}
    \item freely explore morphological properties,
    \item reduce the amount of time required between an idea and its experimentation on an actual robotic platform in the real world,
    \item makes experiments that should be easy to do, actually easy to do,
    \item make the work easily reproducible in any other lab,
    \item keep the work modular and free to use in accordance with open source principles, so it can be reused and extended for other projects.
\end{itemize}

Our approach follows novel design methods for both design and production, for all technological aspects of the robot (i.e. mechanics, actuation, electronics, software, distribution). In particular these methods relies on 3D printing for all mechanical parts, the Arduino electronic architecture for the sensors acquisition, an easy to use Python API called pypot for the control and finally the distribution of all our work under open source licenses.

Using this methodology, we create the Poppy Humanoid robot, a fully modular robot allowing exploring freely the role of morphology and adapting its body to specific experimental setup. This robot has been released under open source license and all files are easily accessible on the GitHub repository: \url{https://www.github.com/poppy_project/}.

We experiment the use of this robot for several applications. First, as a scientific tool and we show that Poppy can be easily and quickly modified to either explore the role of morphology or to be adapted to different experimental setups.
Based on this work, but from another perspective we investigate the potential impact of such platform for educational and artistic applications.

\textbf{Keywords: Robotic, Embodiment, Morphology, Humanoid, Reproducibility, Open Science, Open Hardware.} 

\textbf{This work has been supported by INRIA and the ERC grant EXPLORERS 24007.}

