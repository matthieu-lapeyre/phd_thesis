% !TEX root = ../thesis.tex
%
\pdfbookmark[0]{Acknowledgement}{Acknowledgement}
\chapter*{Acknowledgement}
\label{sec:acknowledgement}
\vspace*{-10mm}


First of all, I would like to thank Pierre-Yves Oudeyer for supervising me along this Ph.D. thesis.  Apart from his excellent advice and open minded supervision, he has been an example both as a human being and as a professional researcher. Moreover, he strongly supported the development of the Poppy project and is a main motor for the dissemination of the platform.

Numerous thanks for Pierre Rouanet, Jonathan Grizou, Didier Roy, Steve N'Guyen, Nicolas Rabault, Clement Moulin-Frier, Stephane Ribas, and Patrick Guillaud who actively participated to the technological development of the Poppy platform and its open source diffusion. I would like to thank all members of the Flowers laboratory for the amazing and unique atmosphere of the team, for sharing their knowledge and allowing me to learn every day. I also thank our team assistants Nicolas, Nathalie, Catherine, and the numerous internship students. In addition, I would like to thank Jasmine Garside for her amazing work and help as a first proofreader of this manuscript.

Many thanks to the members of my jury, Dr Fethi Ben Ouezdou, Dr Jacques Droulez, Dr Thierry Vieilleville and Mrs Anne-cecile Worms for accepting to review this work and taking part of the PhD defense~\footnote{video of the PhD defense: \url{https://www.youtube.com/watch?v=6f0D-HHyqho}}.

I would like to thank the Inria staff for their support during the Poppy Project birth. IT people, Aurelien Dumez, and Nicolas Sulek, for their work to set up all web platforms, needed to the creation of a successful open source project. General services people, J-P.Brumaud, P.Dieudoné, L.Dufayet for their daily support. The communication service, S.Valerius and L.Kovacic for their excellent advice and work to make the Poppy project known and understood outside the laboratory. The technology transfer service, P.Moussier and M.Cromer for their work on the trademark protection.
Finally, I would like to thanks the Inria general direction for supporting and promoting the project, making this possible the continuation of this Ph.D. thesis work as an applicative engineering project for research and education people.


During these three years, I had the chance to engage in different collaborative works. I would like to thank everyone who takes the risk to start unique projects with a completely new and under development robotic platform. Thus, I thank the LPPA Laboratory for being the very first external entity to build a Poppy Humanoid at the early stage of the development process (only 5 months).
I thank the Comacina Capsule creative Association, A.Braconnier, and M-A.Villard, for their work as the first artistic project with the Poppy platform. I thank the Citée des Sciences museum for the autonomous organization of Hackathon dedicated to our project. I thank the Saintonge Sainte-Famille high school and especially J.Claverie for being the first educational establishment to use Poppy as a tool for young students. I thank the ENSAM Paris and Bordeaux engineering school for including Poppy in their course.
And finally, I would like to thank all members and actors of the growing Poppy community.


Unlimited thanks to my family for always have been supportive with all the projects I started since I was a young child, and especially to have always allowed me to do what I wanted. Among all, I thank my dear Rafiaa for being here along all these years, her open-minded ideas and the inexhaustible source of inspiration she is to me, also I guess the Poppy project creation would not have been possible without the unlimited support she gave to me.
