% !TEX root = ../thesis.tex
%

{\usekomafont{chapter}Résumé}
\label{cha:abstract-diff} \\


La contribution principale de cette thèse est la conception de la plateforme robotique Poppy qui peut avoir des applications dans les domaines scientifiques, artistiques et pédagogiques. Poppy est une plate-frome robotique communautaire, modulaire et reproductible.

Dans ce manuscrit sera introduit le context et les challenges scientifiques qui ont motivé à concevoir une nouvelle platforme. Ensuite nous presentrons notre approche de conception permettant de librement modifier la morphologie d'un robot et d'assurer sa reproductibilité scientifique. Puis nous décrirons la plate-forme Poppy Humanoid qui s'appuie sur cette méthodologie de conception. Nous montrerons plusieurs experimentations: scientifiques, artistiques et pédagogiques utilisant la Poppy Humanoid.
Enfin nous finirons par plusieurs discussions.


\section*{Introduction}


Dans l'équipe flowers nous considérons les robots comme des outils qui permettent de mieux comprendre les mecanismes des êtres vivants. Parmis tous ces mecanismes nous sommes particulièrement intéressés par le developpement cognitif, l'interaction physique et sociale, l'apparition du langage, l'apprentisage de nouvelles compétences sensori-motrices (ex: la marche, l'équilibrage, attrapper des objets), l'auto-organisation, l'émergence de comportement complexes, ...

Un principe essentiel de tous ces mécanismes naturels est leur incarnation dans un corps physique en interaction avec le monde réel.

Une question qui se pose est quel est le role de cette incarnation sur ces mécanismes.

Ces vingt dernières années plusieurs scientifiques ont adressé la question. L'un des plus célèbre est rolf pfeiffer qui a montré que le comportement d'un robot ne dépend pas uniquement de son contrôle mais qu'il est aussi necessaire d'integrer la morphologie du robot. Un autre exemple très impressionnant est le travail de Tad McGeer autour de la marche passive. Dans son “robot" qui est en fait un structure purement mécanique, sans moteur, ni controlleurs. Pourtant, cette structure est capable de démontrer un comportement de marche dont la dynamique parait très similaire à celle que l'on peut observer chez l'homme. Ce comportement est uniquement résultant des propriétés physiques de la plate-forme qui sont la longueur des segments, la forme des pieds, la position des centre de masses en interaction avec l'environnement qui est constitué de la gravité et d'un plan legèrement incliné. Ces travaux ont été repris plus recemment mais en conservant le même principe des chercheurs de XXX ont réussi à faire marche un robot sur un tapis roulant pendant plus d'un demi heure. Un dernier exemple que l'on peut partager dans ce résumé est celui de l'artiste Theo Jansen. Il pratique la sculpture cinématique et son travail s'intéresse à la création de nouvelles forme de vie artificielles qui seraient capables de "vivre" de façon autonome sur les plages des Pays Bas. Il est créé un mécanisme assez original uniquement composé de tube en plastique et qui ne necessite qu'un degré de liberté faisant se mouvoir toutes les jambes en même temps. Ce mécanisme tire son energie du vent. Il a poussé ensuite le concept jusqu'à inclure des mecanismes permettant à sa structure de changer de direction, stocker l'énergie et detecter la presence d'eau. Tous ces mecanismes sont purement mécaniques et le comportement né de leurs interactions avec l'environnement.


Ces travaux montrent bien que le comportement d'un robot ne dépend pas uniquement du contrôle mais aussi fortement de ses propriétés morphologiques et de l'environnement dans lequel il agit.

Cependant une limite dans la pratique scientifique de ces travaux et même plus généralement en robotique est que les résultats dépendent donc fortement de la plate-forme et de sa morphologie mais que les plate-formes robotiques ne sont quasiment jamais diffusés avec les papiers scientifiques ce qui pose en problème en ce qui concerne la reproductibilité scientifique.


Néanmoins, lorsque l'on s'interesse à l'étude du rôle de l'incarnation et des propriétés morphologiques sur le comportement, il est necessaire d'avoir de travailler avec des plate-formes robotique.

Comme nous l'avons vu, certains prototypes paraissent très interessant pour cet usage mais ne sont malheuresement pas reproductible: Leurs plans de conception n'est pas distribué et leur morphologies sont souvent complexes avec des propriétés difficiles à reproduire.

Si l'on veut des plate-formes qui soient reproductibles dans plusieurs laboratoires, actuellement il n'y a principalement que des plate-formes commerciales qui sont conçues pour une production de masse et par consequent reproductibles mais leur conception n'est pas adaptée à l'exploration de la morphologie car une modification de leur structure est dans la pratique impossible (trop complexe ou couteux).

Il donc necessaire de trouver des méthodes alternatives qui permettent à la fois de considérer le corps robotique comme une variable experimentale dans lquel on peut modifier tous les parametres morphologiques. Et en même temps créer des plate-formes qui soient reproductibles dans d'autres laboratoires.

Il existe quelques initiatives qui vont dans ce sens. On peut citer Locokit qui proposent des élements basiques que l'on peut assembler pour réaliser des structures mobiles dans lesquelles on peut modifier certains paramètres comme la longueur des segments ou la position des centres de masses.
Une des plate-formes les plus connues et utilisées est Icub qui a l'avantage d'avoir été reproduit en plusieurs exemplaire, d'être partagé entre different laboratoires permettant la reproduction de certaines experiences dans plusieurs laboratoires. De plus, comme cette plate-forme est open source, on pourrait en soit, modifier l'integralité du robot pour explorer differentes morphologies. Cependant les méthodes de conceptions de cette plate-forme rendent complexes et couteux sa modification. En pratique, aucun labo n'a modifié la strucutre de ce robot et seul le laboratoire d'origine travaille sur l'évolution mécanique de l'iCub.


Nous avons donc décidé de lancer le projet Poppy qui a pour but la création d'une plate-forme robotique modulaire, accessible, reproductible et communautaire.

\section*{Le projet Poppy}
\label{sec:Le projet Poppy}
